Rockart\footnote{\cite{Rockart.1979}} entwickelte als einer der Ersten den Ansatz, Erfolgsfaktoren zu identifizieren und Unternehmensleistung zu messen.\footnote{Vgl. für diesen und den folgenden Satz \cite{Chow.2008} S.962.}
Bullen und Rockart\footnote{\cite{Bullen.1981}} sowie Rockart und Crescenzi\footnote{\cite{Rockart.1984}} etabliereten und verfeinerten den Ansatz daraufhin.\\
Seit Beginn der Forschung zu Erfolgsfaktoren von Projekten, wurden immer wieder Auflistungen von Faktoren 
geliefert, jedoch wurde nur eine geringe Priorität auf das Klassifizieren von EF gelegt.\\\noindent
Der Vorteil einer solchen Einordnung in Gruppen liegt darin, dass man, anstatt einzelne Erfolgsfaktoren zu analysieren, zunächst die Gruppen identifizieren kann, in die die Faktoren
einzuordnen sind.\footnote{Vgl. zu diesem und dem folgenden Satz \cite{Belassi.1996}, S. 142.} Daraufhin können dann die kombinierten Auswirkungen der Faktoren auf den Projekterfolg zu bestimmen.

Die Klassifizierung von EF kann jedoch bei der Analyse der Wechselwirkung zwischen diesen und deren Konsequenzen helfen.\footnote{Vgl. \cite{Belassi.1996} S.142.} 



%\textit{Motivation:}\\\noindent
%Die Motivation und das formulierte Problem aus Sicht der Autoren liegt in der Ambiguität der Beurteilung von Projekterfolg.S141-142
%Für diese sehen die Autoren zwei Gründe: Der erste und untergeordnete Grund basiert auf der subjektiven Einschätzung des 
%Projektergebnisses. Auch Projekte, die das Management als gescheitert ansieht, können für den Kunden ein Erfolg sein (ZITAT)141.
%Der primäre Problemansatz liegt aber in der starken Variation der Auflistungen von EF in der Fachliteratur. Dies begünstigt (??) den 
%generellen Trend zur Einordnung von EF in Tabellen, anstatt diese strukturiert bestimmten Kriterien zuzuordnen und so zu kategorisieren. 
%Weitgehend sind es oftmals nicht nur die einzelnen EF, die einen großen Einfluss auf den Erfolg eines Projektes haben, sondern auch das Zusammenspiel 
%von mehreren Faktoren aus unterschiedlichen Kategorien und in unterschiedlichen Phasen des Projektes.\\