Erfolgsfaktoren im Alllgemeinen sind eine limitierte Anzahl von Faktoren, deren zufriedenstellendes Ergebnis erfolgreiche und 
wettbewerbsfähige Leistung für einzelne Bereiche oder das ganze Unternehmen sicherstellen.\footnote{Vgl. \cite{Bullen.1981} S.7.}
Bezogen auf Projekte sind dies Schlüsselfaktoren, die den Erfolg des Projektes fördern,\footnote{Vgl. \cite{Buschermohle.2010} zitiert nach \cite{Basten.2012}.} wie
z.B. Führungskompetenz und Erfahrung des Projektleiters, Kommunikation im Team oder Unterstützung des Managements.\\
Oft wird auch von kritischen Erfolgsfaktoren gesprochen, wobei es für diesen Begriff keine allgemein anerkannte Definition gibt.\footnote{Vlg. \cite{Basten.2012}, S. 59.}