Bereits seit mehreren Jahrzehnten werden die Faktoren, welche den Erfolg von IS-\-Pro\-jek\-ten positiv beeinflussen, untersucht.\todo{zitat evtl weg}\footnote{fortune white 2006}
%So untersuchte Daniel\footnote{\cite{Daniel.1961}} bereits 1961 den Einfluss von Erfolgsfaktoren\todo{\ldots}
%Dabei wurden unterschiedliche Arten gefunden, diese \ac{EF} in Gruppen einzuordnen und somit zu klassifizieren.
%Daraus ergibt sich ein theoretisches Erkenntnissdefizit, da es keine einheitliche, anerkannte Liste von \ac{EF} gibt.\footnote{Vgl. \cite{vanScoter.2011}, S.6.} 
%Dies kann wiederum dazu führen, dass die Häufigkeit, mit der Projekte scheitern oder aus dem Zeit- bzw. Kostenrahmen laufen, 
%steigt, da den Projektverantwortlichen die \ac{EF} nicht ausreichend bekannt sind.\footnote{Vgl. \cite{Sudhakar.2012}, S. 538.}\todo{sinn rein bringen!}
%Ein Problem für die Forschung ergibt sich aus der, in der Fachliteratur fehlenden, umfassenden Übersicht über diese Klassifizierungsarten, die die verschiedenen Autoren liefern.
%Dieser Mangel erschwert die strukturierte Erfassung der maßgeblichen Komponenten, die das Gelingen eines IS-Projektes positiv beeinflussen und begünstigt somit die Wahrscheinlichkeit des Scheiterns dieser Projekte.\todo{Zitat}
%Der Mehrwert einer solchen Zusammenfassung besteht in der Möglichkeit, die Ergebnisse einer retrospektiven Analyse und die Gründe des Projekterfolgs leichter einzuordnen.
Durch das größer werdende Interesse an ebenjenen Faktoren, die den Erfolg eines (IS-)Projektes maßgeblich 
beeinflussen können, wurden im Laufe der Jahre zahlreiche Arten gefunden, Erfolgsfaktoren in Gruppen einzuordnen und folglich 
zu klassifizieren. %Dies führte dazu, dass es bis heute keine allgemeingültige Klassifizierungsart von Erfolgsfaktoren gibt.
Diese Vielzahl an Klassifizierungsarten birgt jedoch das Risiko, dass den Verantwortlichen 
eines Projektes die für dieses Projekt spezifischen Erfolgsfaktoren unbekannt sind, 
oder sie den Faktoren keine angemessene Gewichtung zuordnen 
können. Das Fehlen einer Übersicht, bzw. der Überfluss an EF und ihren Klassifizierungsarten kann zu einer gesteigerten 
Häufigkeit von scheiternden Projekten führen, das für die jeweiligen Unternehmen nicht nur finanziell sehr teuer sein kann.
Oft geht auch ein erheblicher Imageverlust mit scheiternden Projekten einher. 
Hier liegt demnach ein theoretisches Erkenntnisdefizit vor, da es keine einheitliche, anerkannte Übersicht von 
Klassifizierungsarten von Erfolgsfaktoren gibt. Weil somit ein Vergleich der Klassifizierungsarten sehr erschwert wird, stellt dieser
Mangel ebenso ein Problem für die Forschung dar, da hierdurch die strukturierte Erfassung der wichtigsten 
Komponenten für (IS-)Projekterfolg und deren Hierarchisierung entgegengewirkt wird. Die begründet sich 
darin, dass die Verfasser der Klassifizierungsarten sich zwar teilweise aufeinander beziehen (ZITAT LISTE), die 
Hypothese eines anderen jedoch oftmals nicht stützen, sondern eine neue Klassifizierungsart entwickeln.\\\noindent
Der Mehrwert einer Übersicht über die Klassifizierungsarten von Erfolgsfaktoren besteht in der Möglichkeit, 
die verschiedenen Arten und ihre Verbreitung quantifizierbar festzuhalten und, durch Beseitigung der wenig verwendeten 
Klassifizierungsarten, die Artenvielfalt möglicherweise zu dezimieren. Dies würde Projektleitern die Ermittlung der für 
sie relevanten Klassifizierungsart und folglich deren Erfolgsfaktoren erleichtern und es ihnen ermöglichen, die Ergebnisse einer 
retrospektiven Analyse und die Gründe des Projekterfolges übersichtlich einzuordnen.
\\Zur Erstellung einer oben beschriebenen Übersicht muss man sich die folgende Frage stellen:
\textit{Welche unterschiedlichen Klassifizierungsarten von Erfolgsfaktoren von IS-Projekten werden in der Fachliteratur aufgezeigt und wie werden die Ansätze begründet?}\\
Hieraus definiert sich das Forschungsproblem:
Die weitere Forschung im Bereich der Erfolgsfaktorenklassifizierung gestaltet sich schwierig, da
verschiedene, uneinheitliche Ansätze der Klassifizierung von Erfolgsfaktoren existieren und es darüber hinaus keine Übersicht über die bereits \todo{Nomen, vorhandenen klein} Vorhandenen gibt.
Ein systematisches Literraturreview kann dazu beitragen, die bereits bestehenden Klassifizierungen zu extrahieren 
und eventuell Defizite oder Forschungslücken aufzeigen.
Eine aus diesem Review resultierende Übersicht über die in der Fachliteratur aufgeführten \KAS von 
Erfolgsfaktoren in IS-Projekten würde zur Lösung des Forschungsproblems beitragen.