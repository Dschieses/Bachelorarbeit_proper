Um einen Überblick über die in der Fachliteratur verwendeten Klassifikationen von Erfolgsfaktoren in IS-Projekten zu schaffen, wird ein 
systematisches Literaturreview durchgeführt. Dazu werden die Online-Portale von ACM digital library, AIS Electronic Library (AISeL), 
EBSCOhost("`Academic Search Complete” und "`Business Source Complete"), EmeraldInsight, IEEEXplore, ProQuest, ScienceDirect, SpringerLink und 
Wiley InterScience nach relevanten Texten durchsucht. Die Titel, Schlagwörter oder Zusammenfassungen der Literatur
sollen eine logische Verknüpfung der folgenden Begriffe enthalten: Erfolgsfaktor, Projekt, Software, Informationssystem und Informationstechnologie. 
Dabei muss darauf geachtet werden, dass sowohl die verschiedenen englischen Schreibweisen der Begriffe abgedeckt werden, als auch, dass „Success Factor“ und „Project“ und
mindestens einer der Begriffe „Software“, “Information System“ oder „Information Technology“ enthalten ist.\\ 
Die gefundenen Ergebnisse werden dann in einer Tabelle redundanzfrei festgehalten. Beim Lesen der Abstracts und gegebenenfalls der Texte wird festgestellt, ob diese sich tatsächlich auf den 
gewünschten Sachverhalt beziehen. Suchergebnisse, die in keinerlei inhaltlichem Zusammenhang zur Thematik stehen, werden hierbei verworfen.
Sollte in diesen Texten auf noch nicht berücksichtigte, relevante Literatur verwiesen werden, so wird diese nachgetragen.
Darauf folgt ein intensives Studium der relevanten Texte, in welchem die von den Autoren aufgezeigten Klassifizierungsarten und Begründungen extrahiert werden.


