Um einen Überblick über die in der Fachliteratur verwendeten Klassifikationen von \EF in IS-Projekten zu schaffen, wurde ein 
systematisches Literaturreview in den Datenbanken von AIS Electronic Library (AISeL), 
EBSCOhost("`Academic Search Complete” und "`Business Source Complete"), ProQuest und ScienceDirect durchgeführt.
Dabei wurden jeweils die Titel, Schlagwörter und Abstracts nach den Begriffen 
"`Erfolgsfaktor"', "`Projekt"' und "`Informationssystem"', welche mit einem logischen UND verknüpft waren, durchsucht. 
Hierbei mussten die verschiedenen englischen Schreibweisen der Begriffe verwendet, sowie Plural und Synonyme abgedeckt werden.
Zum Beispiel wurden als Synonyme für Informationssystem auch die Begriffe "`Software"' und "`Informationstechnologie"' bzw. "`Information Technology"' verwendet.\\
Die gefundenen Ergebnisse wurden dann in einer Tabelle redundanzfrei festgehalten und einem weiteren Auswahlprozess unterzogen:
Nur Literatur, die sich schwerpunktmäßig mit \EF und deren Klassifizierung beschäftigt, sollte weiter betrachtet werden.
Dies wurde durch das Lesen des Abstracts festgestellt. Bei Unklarheiten wurden zusätzlich einzelne Textabschnitte,
vorrangig z.B. die Einleitung oder das Fazit, berücksichtigt.
Darauf folgte ein intensives Studium der verbliebenen Texte, in welchem die von den Autoren aufgezeigten Klassifizierungsarten und Begründungen extrahiert wurden.
\todo{In diesen Texten zitierte, relevante, jedoch noch nicht berücksichtigte Literatur wurde zusätzlich in das Studium aufgenommen.}