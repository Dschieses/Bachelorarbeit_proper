\noindent
\subsubsection{Klassifizierung nach Salmeron und Herrero (2005)}
\textit{Vorgehehen:}\\\noindent
In ihrer Studie beziehen sich Salmeron und Herrero auf die Erfolgsfaktoren %bei der Implementierung
von \textit{Executive Information Systems}, also 
Führungsinformationssystemen (\acs{FIS}) oder Management-Unterstützungs-Systemen (\acs{MUS}).
\footnote{Vgl. zu diesem und dem Folgenden Satz \cite{Salmeron.2005}, S. 1.}
Unter Nutzung des Analytischen Hierarchie Prozesses (\acs{AHP}) liefern sie eine Priorisierung der Erfolgsfaktoren, die sie als kritische \ac{EF} definieren.
Der \ac{AHP} ist eine starke und flexible Methode, die von dem Mathematiker Thomas Saaty entwickelt wurde,\footnote{Zur Entwicklung des AHP siehe \cite{Saaty.1977} sowie \cite{Saaty.1980}} 
um Entscheidungsprozesse zu unterstützen und Prioritäten zwischen verschiedenen Attributen zu setzen.\footnote{Vgl. zu diesem Satz und dem Rest des Absatzes \cite{Salmeron.2005}, S. 3.}
Dazu muss in einem ersten Schritt das Entscheidungsproblem in eine Hierarchie von zusammenhängenden Elementen aufgeschlüsselt werden.
Darauf folgen im zweiten Schritt paarweise Vergleiche der Elemente und Berechnungen der Attributgewichtungen.
In einem letzten Schritt müssen die Gewichtungen der Kategorien berechnet werden.\\
Die Klassifizierung, die Salmeron und Herrero aus der Verwendung des \ac{AHP} ableiten, 
unterteilt die Erfolgsfaktoren in folgende Gruppen:\footnote{Vgl. zu dieser Liste \cite{Salmeron.2005}, S. 4.}
\begin{itemize}\itemsep0pt
\item[-]Personelle Ressourcen,
\item[-]Informationen und Technologie,
\item[-]System Wechelwirkungen.
\end{itemize}
\noindent
Zur Faktorgruppe \textit{Personelle Ressorcen} zählen Sie die Faktoren \textit{Benutzereinbeziehung},die \textit{Existenz von Unterstützung des Auftraggebers}
sowie die \textit{Notwendigkeit von kompetentem und ausgewogenem Personal}.\footnote{Vgl. \cite{Salmeron.2005}, S. 4.}
Die Kategorie \textit{Informationen und Technologie} beinhaltet, laut Salmeron und Herrero, \textit{geeignete Hardware und Software} sowie \textit{richtige Informationen}.\footnote{Vgl. zu diesem und dem folgenden Satz \cite{Salmeron.2005}, S. 5.}
Zur letzten Gruppe, den \textit{System Wechselwirkungen} zählen sie die Faktoren \textit{schnelle Prototypentwicklung}, \textit{maßgeschneiderte Systemlösungen} sowie ein \textit{flexibles und sensitives System}.
\\\noindent\textit{Begründung:}\\\noindent