\noindent
\subsubsection{Klassifizierung nach Sudhakar (2012)}
\textit{Vorgehehen:}\\\noindent
Sudhakar stellt in seiner Studie folgende Kategorien vor, in die sich die Faktoren einordnen lassen, die er bei einem gründlichen
Literaturreview gefunden hat:\footnote{Vgl. zu dieser Liste \cite{Sudhakar.2012}, S. 545.}
\begin{itemize}\itemsep0pt
\item[-]Kommunikationsfaktoren,
\item[-]Technische Faktoren,
\item[-]Unternehmensfaktoren,
\item[-]{Umweltfaktoren,}
\item[-]{Produktfaktoren,}
\item[-]{Teamfaktoren,}
\item[-]{Projektmanagementfaktoren.}
\end{itemize}
\textit{Begründung:}\\\noindent
Sudhakar führt an, dass sich seine Forschung sowohl ergänzend auf die Literatur, als auch auf die Praxis auswirkt.\footnote{Vgl. zu diesem Absatz \cite{Sudhakar.2012}, S. 553.}
Auf Seiten der Literatur wirkt seine Forschung dahingehend, dass sie dem Literaturbestand den wichtigen Aspekt des 
konzeptionellen Modells von Erfolgsfaktoren verschiedener Dimensionen hinzufügt.
In der Praxis können Projektmanager von den gefundenen Erfolgsfaktoren und deren Kategorisierung profitieren.
Somit könne weiterhin auf das höchste Ziel des Projektmanagements,den Erfolg des Projekts, hingearbeitet werden.

%Sudhakar begründet seine Klassifizierung zunächst mit der Feststellung, dass dieser Ansatz auf Projekte in anderen Branchen und 
%Ländern adaptiert werden kann.\footnote{Vgl. zu diesem Absatz \cite{Sudhakar.2012}, S. 554.}
%Darüber hinaus kann, laut Sudhakar, sein Ansatz auf multinationale Projekte, die mehrere Branchen oder Unternehmen involvieren, 
%ausgeweitet werden.


%Dies führte global gesehen zu einer gesteigerten Erfolgsrate von Projekten.