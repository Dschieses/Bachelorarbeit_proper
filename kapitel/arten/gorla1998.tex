\noindent
\subsubsection{Klassifizierung nach Gorla und Lin (1998)}
\textit{Vorgehen:}\\\noindent
Gorla und Lin nehmen Bezug auf den Erfolg von Management-Informationssystemen (\acs{MIS}).\footnote{Vgl. \cite{Gorla.1998}, S. 728.}
Ihrer Ansicht nach beeinflussen Faktoren folgender Gruppen den Erfolg von \ac{MIS}.\footnote{Vgl. zu dieser Liste \cite{Gorla.1998}, S. 728-729.}
\begin{itemize}\itemsep0pt
\item[-]Faktoren bezüglich der Organisation,
\item[-]technische Faktoren,
\item[-]individuelle Faktoren.
\end{itemize}
\textit{Begründung:}\\\noindent
Gorla und Lin führen an, dass mit ihrem Modell der Erfolgs von MIS-Projekten vorhergesagt werden kann.\footnote{Vgl. zu diesem Satz und dem Rest des Absatzes \cite{Gorla.1998}, S. 729.}
Ihr Ansatz definiere, im Gegensatz zu früheren Studien, den Erfolg von MIS-Projekten empirisch durch die Nutzung von Pseudo-Variablen.
Dadurch, dass sie sowohl technische als auch organisatirische Faktoren betrachten, sei ihr Modell umfassender als Bisherige.
