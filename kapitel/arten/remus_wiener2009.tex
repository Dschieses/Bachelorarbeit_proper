\noindent
\subsubsection{Klassifizierung nach Remus und Wiener (2009)}
\textit{Vorgehehen:}\\\noindent
Remus und Wiener beziehen sich in ihrer Studie auf die Erfolgsfaktoren für das Management von Offshore-Softwareentwicklung.\footnote{Vgl. \cite{Remus.2009}, S. 6.}
\textit{Offshoring} bedeutet hierbei, dass der komplette Softwareentwicklungsprozess oder Teile dessen in das Ausland verlagert werden,
da dort die Kosten sehr viel niedriger als im Inland sind und dort mehr Arbeitskräfte zur Verfügung stehen.\footnote{Vgl. \cite{Krishna.2004}, S. 62.}
\ldots
Sie unterscheiden dabei in die Dimensionen \textit{intern versus extern} sowie \textit{statisch versus dynamisch}.\footnote{Vgl. zu diesem und dem Folgenden Satz und der Liste \cite{Remus.2009}, S. 13.} 
Aus der Kombination dieser Dimensionen ergeben sich die folgenden Gruppen:
\begin{itemize}\itemsep0pt
\item[-]Interne Eignungsfaktoren,
\item[-]interne Managementfaktoren,
\item[-]externe Eignungsfaktoren,
\item[-]externe Managementfaktoren.
\end{itemize}
\textit{Interne Eignungsfaktoren} beziehen sich auf die Bereitschaft zum \textit{Offshoring} auf Seiten des Clienten.
Zu dieser Gruppe gehören zum Beispiel \textit{internationale Unternehmenskultur} sowie \textit{standardisierte und dokumentiere Prozesse}
\textit{Interne Managementfaktoren} beziehen sich auf die Planung von \textit{Offshore-Softwarentwicklung}.
Dieser Gruppe enthält Faktoren wie die \textit{Definition von klaren Projektzielen} und die \textit{Entwicklung eines umfassenden Business Cases}.
\textit{Externe Eignungsfaktoren} beziehen sich auf die Auswahl eines Offshore-Privders.
Hierzu zählen Remus und Wiener zum Beispiel die \textit{rechtliche und politische Stabilität im Offshore-Land} und die \textit{hohe Qualität der Offshore-Arbeitnehmer}.
\textit{Externe Managementfaktoren} beziehen sich auf die Durchführung von \textit{Offshore-Softwarentwicklung}.
Zu dieser Gruppe zählen zum Beispiel die \textit{kontinuierliche Kontrolle der Projektergebnisse} sowie die \textit{Definition eines korrekten Vertrags}.
\\\noindent
\textit{Begründung:}\\\noindent