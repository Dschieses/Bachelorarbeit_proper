\noindent
\subsubsection{Klassifizierung nach Belassi und Tukel (1996)}
\textit{Vorgehen:}\\\noindent
Belassi und Tukel beziehen sich in ihrer Studie von 1996 auf Projektmanagement im Allgemeinen und nicht spezifisch auf
Erfolgsfaktoren von IS-Projekten.\footnote{Vgl. \cite{vanScoter.2011}, S. 4.}
Das Schema, welches Belassi und Tukel zur Klassifizierung von Erfolgsfaktoren anwenden, beinhaltet vier Kategorien:\footnote{Vgl. zu dieser Liste \cite{Belassi.1996}, S. 143-145.}
\begin{itemize}\itemsep0pt
\item[-]Faktoren bezogen auf die Eigenschaften des Projektes, wie Größe, Dauer und Projektnetzwerk,
\item[-]Faktoren bzgl. des Projektmanagements und der Teammitgliedern, in ihrer Wichtigkeit abhängig von der Phase des Projektes,
\item[-]Faktoren bzgl. der Organisation/des Unternehmens, hier ist als wichtigster Faktor die Unterstützung durch das Top-Management zu erwähnen,\footnote{Vgl. zum zweiten Halbsatz \cite{Belassi.1996}, S. 146.}
\item[-]Faktoren die außerhalb des Unternehmens liegen, aber dennoch Einfluss auf das Projekt haben wie zum Beispiel politische, ökonomische oder soziale Faktoren.
\end{itemize}
\textit{Begründung:}\\\noindent
Ziel von Belassi und Tukel war es, mit ihrer Studie eine Möglichkeit aufzuzeigen, Erfolgsfaktoren zu klassifizieren und deren 
Einfluss auf die Projektperformance zu identifizieren.\footnote{Vgl. \cite{Belassi.1996}, S. 142.}
Sie begründen, dass der Vorteil des hier aufgezeigten Schemas in der Möglichkeit liegt, die Abhängigkeit zwischen Projekterfolg
und den Faktorgruppen Projektmanager und/oder Projekt und/oder externe Faktoren, leichter aufzuschlüsseln.\footnote{Vgl. zu diesem Satz und dem Rest des Absatzes \cite{Belassi.1996}, S. 143-144.}
Das so entwickelte Rahmenkonzept fördert das Verständnis des Projektmanagers für die Beziehungen der Faktoren zwischen den Gruppen. 
Dies führt dazu, dass der Projektmanager sein Projekt präziser beurteilen und überwachen kann. Zudem ist dieses Konzept sehr leicht an spezifische Projekte anzupassen und kann so einfach erweitert werden.
