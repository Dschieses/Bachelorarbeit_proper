\noindent
\subsubsection{Klassifizierung nach Slevin und Pinto (1987)}
\textit{Vorgehen:}\\\noindent
Slevin und Pinto liefern eine Liste von 10 Erfolgsfaktoren für die erfolgreiche Implementierung von Projekten, welche sie in folgende
Gruppen einordnen:\footnote{Vgl. zu dieser Liste \cite{Slevin.1987}, S. 3.}
\begin{itemize}\itemsep0pt
\item[-]Strategische Faktoren,
\item[-]Taktische Faktoren.
\end{itemize}
Die strategischen Faktoren beziehen sich dabei auf das Aufstellen von übergeordneten Zielen und die Planung, wie diese zu erreichen sind.\footnote{Vgl. zu diesem und dem folgenden Satz \cite{Slevin.1987}, S. 3.}
Zu den taktischen Faktoren zählen Slevin und Pinto die menschlichen, technischen und Finanziellen Ressourcen, die zum erreichen der
strategischen Ziele nötig sind.
\\\noindent\textit{Begründung:}\\\noindent
