\noindent
\subsubsection{Klassifizierung nach Slevin und Pinto (1987)}
\textit{Vorgehen:}\\\noindent
Slevin und Pinto beziehen sich in ihrer Studie auf Erfolgsfaktoren im allgemeinen Projektmanagement.
Sie erstellen eine Liste von zehn Erfolgsfaktoren für die erfolgreiche Implementierung
von Projekten, welche sie zunächst zwei verschiedenen Projektphasen zuordnen.\footnote{Vgl. zu diesem und dem folgenden Satz \cite{Slevin.1987}, S. 3.} 
Die Faktoren \textit{Projektrichtung}, \textit{Top-Management-Unterstützung} und \textit{Zeitplan} teilen sie in die Planungsphase eines Projektes ein. 
Die restlichen sieben \EF,wie zum Beispiel Kundenkonsultation, technische Aufgaben oder Akzeptanz des 
Kunden,\footnote{Vgl. zu den Erfolgsfaktoren \cite{Slevin.1987}, S. 2.} ordnen die Autoren der Implementierungs - 
bzw. Aktionsphase zu.\\\noindent
Alle \EF klassifizieren sie schließlich in:\footnote{Vgl. zu dieser Liste \cite{Slevin.1987}, S. 3.}
\begin{itemize}\itemsep0pt
\item[-]Strategische Faktoren,
\item[-]Taktische Faktoren.
\end{itemize}
Die strategischen Faktoren beziehen sich dabei auf das Aufstellen von übergeordneten Zielen und die Planung, wie diese zu erreichen sind.\footnote{Vgl. zu diesem und dem folgenden Satz \cite{Slevin.1987}, S. 3.}
Zu den taktischen Faktoren zählen, laut Slevin und Pinto, die menschlichen, technischen und finanziellen Ressourcen, die zur Erreichen der
strategischen Ziele notwendig sind. Aufbauend auf diesem Konzept entwickeln Slevin und Pinto eine Agenda mit fünf Schritten, die, bezüglich
der Klassen Strategie und Taktik, die Wahrscheinlichkeit einer erfolgreichen Implementierung erhöhen:\footnote{Vgl. zu diesem Satz und der Liste \cite{Slevin.1987}, S. 8-9.}
\begin{itemize}\itemsep0pt
\item[1.]Nutzen eines Multi-Faktor-Modells
\item[2.]Frühes strategisches Denken im Projektlebenszyklus
\item[3.]Im fortschreitenden Lebenszyklus mehr taktisches Denken
\item[4.]Bewusste Planung und Kommunikation des Übergangs von Strategie zu Taktik
\item[5.]Fokus gleichermaßen auf Strategie und Taktik
\end{itemize}
\noindent\textit{Begründung:}\\\noindent
Laut Slevin und Pinto setzt ihr Modell am ersten Schritt der Agenda an, da
es für Projektmanager essentiell sei, mehrere Faktoren gleichzeitig zu berücksichtigen.\footnote{Vgl. zu diesem Satz und dem Rest des Absatzes \cite{Slevin.1987}, S. 8.}
Ihr Studie bietet zehn Erfolgsfaktoren in einem Rahmenkonzept zur Projektimplementierung.
Innerhalb dieses Rahmenkonzeptes werden verschiedenen Faktoren, abhängig von Phase des Projektes, unterschiedliche Relevanz zugeordnet.
Die Priorisierung der \EF erfolgt also sowohl abhängig von zeitlichem Status des Projektes, als auch von der Kategorie des Faktors.

