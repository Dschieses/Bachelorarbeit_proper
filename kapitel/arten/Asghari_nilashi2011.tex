\noindent
\subsubsection{Klassifizierung nach Asgari, Nilashi et al. (2011)}
\textit{Vorgehen:}\\\noindent
In ihrer Studie von 2011 nutzen \ldots den Ansatz der \ac{BSC} um die Erfolgsfaktoren zu klassifizieren. 
Die Balanced Scorecard-Methode ist ein strategischer Ansatz und ein Performance Managementsystem welches die Implementierung
von Visionen und Stragegien ermöglicht und folgende vier Perspektiven unterscheidet:\footnote{Vgl. \cite{Anshari.2011}, S. 3093.}
\begin{imtemize}
\item[-]{Finanzielle Perspektive,}
\item[-]{Benuterperspektive,}
\item[-]{Perspektive der internen Geschäftsprozesse,}
\item[-]{Lern- und Wachstumsperspektive.}
\end{imtemize}\footnote{Vgl. zu dieser Liste \cite{Anshari.2011}, S. 3093.}
Zudem kann die Methode der \ac{BSC} den Managementprozess von den vier verschiedenen Perspektiven beurteilen.\footnote{Vgl. \cite{Anshari.2011}, S. 3093.}
\textit{Begründung:}\\\noindent
noch keine begründung gefunden