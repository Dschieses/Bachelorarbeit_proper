\noindent
\subsubsection{Klassifizierung nach Rockart (1979)}
\textit{Vorgehehen:}\\\noindent
Rockart betrachtet in seinem Aufsatz maßgeblich die "critical success factor method" (Er\-folgs\-fak\-tor-\-Me\-tho\-de), die von einem Forscherteam des \ac{MIT} entwickelt wurde.\footnote{Vgl. zu diesem Teilsatz \cite{Rockart.1979}, S. 82.} Anschließend nimmt er seinerseits eine Unterteilung der  
hier herausgearbeiteten Erfolgsfaktoren von \ISPS in zwei Kategorien vor:\footnote{Vgl. zu der folgenden Liste \cite{Rockart.1979}, S. 92.}
\begin{itemize}\itemsep0pt
\item[-]Faktoren bezüglich der Kontrolle,
\item[-]Faktoren bezüglich der Entwicklung.
\end{itemize}
Darüber hinaus zitiert Rockart das MIT-Team bezüglich der folgenden vier Quellen von Erfolgsfaktoren:\footnote{Vgl. zu diesem Satz und der folgenden Liste \cite{Rockart.1979}, S. 86.}
\begin{itemize}\itemsep0pt
\item[-]Struktur der jeweiligen Branche,
\item[-]Wettbewerbsstrategie, Position innerhalb der Branche und geographische Lage,
\item[-]externe Einflüsse,
\item[-]zeitliche Rahmenbedingungen.
\end{itemize} 
Diese Quellen können, laut Rockart, die Relevanz der einzelnen \EF für die jeweiligen Unternehmen selektiv beeinflussen.\footnote{Vgl. zu diesem Satz \cite{Rockart.1979}, S. 86, 87.}
Je höher beispielsweise der aktuelle Wettbewerbsdruck für ein Unternehmen ist, desto mehr tendiert es dazu, \EF bezüglich der Kontrolle zu 
priorisieren.\footnote{Vgl. zu diesem und dem folgenden Satz \cite{Rockart.1979}, S. 92.}
Je stärker jedoch ein Unternehmen zum Beispiel durch staatliche Vorgaben oder Dezentralisierung isoliert ist, desto mehr fokussiert es auf die zukunftsorientierte 
Entwicklung, die auf eine Anpassung an eine neue Umgebung ausgerichtet ist. Rockart geht schließlich noch einen Schritt weiter und stellt heraus, 
dass \EF zusätzlich nicht nur zwischen Unternehmen, sondern selbst von Manager zu Manager voneinander abweichen können.\footnote{Vgl. \cite{Rockart.1979}, S. 86.} 
\\\noindent
\textit{Begründung:}\\\noindent
Rockart liefert keine explizite Begründung für seine Kategorisierung, geht jedoch in seinem gesamten Artikel auf die
generelle Relevanz von Erfolgsfaktoren ein. Hierzu führt er an, dass z.B. die Identifikation von Erfolgsfaktoren eine präzisere Einschätzung des Umfangs der zu 
beschaffenden Informationen ermöglicht und somit eine teure Mehraquise einschränkt.\footnote{Vgl. \cite{Rockart.1979}, S. 87.} Außerdem betont er die 
Vorteile der von ihm betrachteten Erfolgsfaktor-Methode. Diese baut auf jeweils zwei bis drei Interviews mit Führungskräften aus verschiedenen Unternehmen auf:\footnote{Vgl. zu diesem Abschnitt \cite{Rockart.1979}, S. 84.}
\\\noindent 
Im ersten Interview wurden die Ziele der Interviewten erfasst und die dafür maßgeblichen \EF herausgearbeitet. 
Weiter wurden die Beziehungen zwischen den Faktoren und Zielen beleuchtet und einzelne \EF hinzugefügt, kombiniert oder eliminiert.
Im zweiten Interview wurden die \EF weiter zusammengefasst und präzisiert. Ein drittes Interview zur endgültigen Abrundung der Liste wurde nach 
Bedarf vereinbart.
\\\noindent
Die Vorteile dieses Vorgehens, und des daraus resultierenden Schemas, liegen laut Rockart in der Berücksichtigung von subjektiv und über die Zeit variierenden 
Ansprüchen der Manager an den Informationsgehalt des Modells.\footnote{Vgl. zu diesem und dem nächsten Satz \cite{Rockart.1979}, S. 84.} Ferner dient der Ansatz
dazu, die Führungskräfte zeiteffizient bei der Ermittlung ihres wesentlichen Informationsbedarfes zu unterstützen.