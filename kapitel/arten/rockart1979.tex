\noindent
\subsubsection{Klassifizierung nach Rockart (1979)}
\textit{Vorgehehen:}\\\noindent
Rockart nimmt in seiner Studie eine Unterteilung von Erfolgsfaktoren\todo{welche Projekte etc} in folgende Gruppen vor:\footnote{Vgl. zu dieser Liste \cite{Rockart.1979}, S. 92.}
\begin{itemize}\itemsep0pt
\item[-]Faktoren bezüglich der Kontrolle,
\item[-]Faktoren bezüglich der Erstellung.
\end{itemize}
\textit{Begründung:}\\\noindent
Rockart liefert keine explizite Begründung für seinen Ansatz, geht jedoch in seinem gesamten Artikel auf die
generelle Relevanz von Erfolgsfaktoren ein. 
Hierzu führt er an, dass z.B. die Identifikation von Erfolgsfaktoren eine präzisere Einschätzung
des Umfangs der zu beschaffenden Informationen ermöglicht und somit eine teure Mehraquise einschränkt.\footnote{Vgl. \cite{Rockart.1979}, S. 87.}
