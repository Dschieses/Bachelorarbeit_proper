\subsubsection{Klassifizierung nach Milis und Mercken (2002)}
\textit{Vorgehen:}\\\noindent
In ihrer Studie bezüglich der Einführung von Informations- und Kommunikationstechnologie in belgischen Banken und Versicherungen, 
fanden Milis und Mercken eine große Anzahl von Erfolgsfaktoren. Zur Strukturierung dieser Faktoren entwickelten sie folgendes Rahmenkonzept:
Sie unterteilen grundsätzlich in vier Kategorien:
\begin{itemize}\itemsep0pt
  \item[-]{Faktoren, die Einfluss auf die Zielkongruenz nehmen.}
  \item[-]{Faktoren, die sich auf das Projektteam beziehen.}
  \item[-]{Faktoren, die die Akzeptanz des Projekts und dessen Ergebnis beeinflussen.}
  \item[-]{Faktoren, die sich auf das Vorgehen bei der Implementierung beziehen.}  
\end{itemize}
\textit{Begründung:}\\\noindent
Zusätzlich zu diesem System unterteilung in 8 Kategorien (Tabelle 1)
