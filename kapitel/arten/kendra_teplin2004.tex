\noindent
\subsubsection{Klassifizierung nach Kendra und Taplin (2004)}
\textit{Vorgehen:}\\\noindent
//Fokussiert auf Projektmamagement. Laut Kendra und Taplin müssen Organisationen, um bei der Einführung von neuen Mangementpraktiken erfolgreich zu sein, eine gemeinsame Basis an Werten und Ansichten (Projektmanagementkultur) entwickeln. Der Projektmanagementerfolg begründet sich auf den 4 Dimensionen von Projekterfolg: Fähigkeiten und Kompetenzen  des Projektmanagements (Mikrosozial), Organisatorische Struktur auf Projektebene (Makrosozial), Leistungsmessungssysteme (Mikrotechnisch), Unterstützung durch das Management (Makrotechnisch).
Begründung:
Die von Kendra und Taplin gelieferte Klassifizierung bietet ein Rahmenkonzept, mit der Unternehmen eine Bewertung ihres aktuellen Managementpotentials durchführen können. Hierbei werden die Fähigkeiten des Projektmanagements, die Hauptgeschäftsprozesse, der projektspezifische Ressourcenverbrauch und die nötigen weiteren Schritte, um die Verbesserung des Projektmanagements  zu unterstützen, herausgestellt. Darauf aufbauend kann eine Veränderung in ineffizienten Projektbereichen initiiert werden, sodass die Projektperformance gesteigert wird.

Bezieht EF von anderen Autoren
Interview von IT-leadern 
Unterschieden wird nach zwei Hauptkategorien: Sozial und Technisch. Eine spezifischere Unterteilung in Makrosoziale/-technische und Mikrosoziale/-technische Faktoren liefert die endgültige Klassifizierung.
Kategorisierung:
\begin{itemize}
\item[-]Mikrosozial: Fähigkeiten und Kompetenzen  des Projektmanagements 
\item[-]Makrosozial: Organisatorische Struktur auf Projektebene
\item[-]Mikrotechnisch: stellt ein Leistungsmessungssystem dar - individuelle Maßstäbe, um Performance im organisatorischen und projektbezogenen Bereich zu überwachen
\item[-]Makrotechnisch: Unterstützung durch das Management, Bilden von strukturierten Geschäftsprozessen und -rahmenkonzepten 
\end{itemize}
\textit{Begründung:}\\\noindent
Wobei EF wie Team-Performance und Kundenzufriedenheit in den Mikrotechnischen Bereich fallen

Scio-technical system theory (taylor und felton 1993)
Open system theory (emery und Purser 1996)
