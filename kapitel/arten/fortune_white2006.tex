\noindent
\subsubsection{Klassifizierung nach Fortune und White (2006)}
\textit{Vorgehen:}\\\noindent
Einen weiteren Ansatz zur Klassifizierung liefern Fortune und White. Sie beschreiben in ihrer Studie die Verwendung des 
\ac{FSM},\footnote{Zur Beschreibung des FSM siehe \cite{White.2009}.} welches ursprünglich von Bignell und Fortune entwickelt wurde.\footnote{Vgl. \cite{Fortune.2006}, S. .}
Das FSM ist ein Systemmodell,\ldots
%Das formale System im Kern des Modells umfasst ein Subsystem zur Entscheidungsfindung, ein Subsystem zur Leistungsüberwachung
%sowie eine Reihe von Subsystemen und Elementen die die Aufgaben des Systems und folglich seine Umwandlungen von Inputs in
%Outputs durchführen. 
Basierend auf den Komponenten des \ac{FSM} kommen Fortune und White zu folgender Klassifizierung:\footnote{Vgl. zu dieser Liste \cite{Fortune.2006}, S. 57.}
\begin{itemize}\itemsep0pt
\item[-]{Ziele und Vorgaben}
\item[-]{Leistungsüberwachung}
\item[-]{Entscheidungsträger}
\item[-]{Transformationen}
\item[-]{Kommunikation}
\item[-]{Umgebung}
\item[-]{Beschränkungen}
\item[-]{Ressourcen}
\item[-]{Kontinuität}
\end{itemize}
\textit{Begründung:}\\\noindent
Als Begründung zeigen Fortune und White zunächst zwei Kritikpunkte an dem bestehenden Ansatz\todo{was ist der bestehende ansatz} von Erfolgsfaktoren auf, denen ihre Klassifizierung entgegen wirken soll.\\\noindent
Sie bemängeln als Erstes, dass bei der Auflistung von einzelnen Faktoren nicht auf die Beziehung zwischen diesen eingegangen wird,
dies sei jedoch so wichtig, wie die Faktoren selbst.\footnote{Vgl. \cite{Fortune.2006}, S. 54.}
Gegen diesen Punkt führen sie an, dass das \ac{FSM} genauso stark auf die Beziehung zwischen den Faktoren, wie auf die Faktoren selbst fokussiert.
Als zweiten Punkt kritisieren sie nach Larsen und Myers, dass der Ansatz der Faktoren dazu neigt, die Implementation als
statischen Prozess, anstelle eines dynamischen Phänomens zu sehen.\footnote{Vgl. \cite{Larsen.1999}, S. 398 sowie \cite{Fortune.2006}, S. 54.}
Bezüglich dieses Kritikpunktes argumentieren sie, dass das \ac{FSM} ein dynamisches Modell eines Systems ist, welches auf Entscheidungsfindung reagiert und mit seiner
Umwelt interagiert.\footnote{Vgl. \cite{Fortune.2006}, S. 57.}
