\noindent
\subsubsection{Klassifizierung nach Daniel (1961)}
\textit{Vorgehehen:}\\\noindent
???Daniel liefert keine Klassifizierung von Erfolgsfaktoren, da es den Begriff zu diesem Zeitpunkt noch gar nicht gibt???
Daniel ist einer der Ersten, der überhaupt von Erfolgsfaktoren im Bezug auf Informationssysteme spricht.\footnote{Vgl. \cite{Daniel.1961}, S. 116.}
Er unterteilt diese dabei in die folgenden zwei Kategorien:\footnote{Vgl. zu dieser Liste \cite{Daniel.1961}, S. 113, 114, 117.}
\begin{itemize}\itemsep0pt
\item[-]Planung,
\item[-]Kontrolle.
\end{itemize}
Planung definiert Daniel hier als Zielsetzung, Strategieformulierung und Entscheidungsfindung zwischen alternativen
Anlagen oder Handlungsoptionen.\footnote{Vgl. \cite{Daniel.1961}, S. 113.}
Innerhalb der Gruppe Planung unterteilt er auf einer weiteren Ebene:\footnote{Vgl. zu dieser Liste \cite{Daniel.1961}, S. 113-116.}
\begin{itemize}\itemsep0pt
\item[-]Umweltinformationen, wie soziale, politische oder ökonomische Aspekte,
\item[-]Wettbewerbsdaten, wie vergangene Wertentwicklung oder Programme und Pläne der Wettbewerber,
\item[-]{interne Daten, wie Stärken und Schwächen des eigenen Unternehmens.}
\end{itemize}
Auf Seiten der Kontrollfaktoren gruppiert er noch in finanzielle und nicht-finanzielle Daten.\todo{mehr}
\\\noindent
\textit{Begründung:}\\\noindent
Daniel liefert keine gesonderte Begründung für seinen Ansatz. Er ist jedoch einer der Ersten, der von Erfolgsfaktoren im Bezug auf 
Informationssysteme spricht.
Daniel argumentiert, dass das Problem der hohen Anzahl von scheiternden Projekten in der Lücke zwischen statischen \todo{besser}Informationssystemen(Buchführungssystemen) und
sich wandelnden Unternehmensstrukturen liegt.\footnote{Vgl. \cite{Daniel.1961}, S. 111.}