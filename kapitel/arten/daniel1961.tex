\noindent
\subsubsection{Klassifizierung nach Daniel (1961)}
\textit{Vorgehehen:}\\\noindent
Daniel erwähnt als einer der Ersten überhaupt den Begriff Erfolgsfaktor generell, aber auch im Zusammenhang mit Informationssystemen.\footnote{Vgl. \cite{Daniel.1961}, S. 116 sowie \cite{Fortune.2006}, S. 53 und \cite{Rockart.1979}, S. 85.}
Ihm zufolge gibt es in den meisten Branchen drei bis sechs Faktoren, denen besondere Aufmerksamkeit geschenkt werden muss,
um eine Unternehmung positiv zu beeinflussen, da sie den Erfolg maßgeblich bedingen.\footnote{Vgl. \cite{Daniel.1961}, S. 116.}
Prinzipiell beabsichtigt Daniel mit seinem Artikel in erster Linie, die Relevanz eines eigenständigen Informationssystems 
herauszustellen.\footnote{Vgl. zu diesem und dem folgenden Satz \cite{Daniel.1961}, S. 113.}
Seiner Ansicht nach liegt der Schlüssel zu einem dynamischen und nützlichen System in zwei entscheidenden Elementen des Managementprozesses:
\begin{itemize}\itemsep0pt
\item[-]Planung und
\item[-]Kontrolle,\footnote{Vgl. zu dieser Liste \cite{Daniel.1961}, S. 113, 114, 117.}
\end{itemize}
wobei Daniel primär auf den Aspekt der \textit{Planung} fokussiert.\footnote{Vgl. zu diesem Teilsatz \cite{Daniel.1961}, S. 113.} 
\textit{Planung} definiert er im Folgenden als Zielsetzung, Strategieformulierung und Entscheidungsfindung zwischen alternativen
Anlagen oder Handlungsoptionen,\footnote{Vgl. zu diesem Teilsatz \cite{Daniel.1961}, S. 113.} während er \textit{Kontrolle} nicht gesondert definiert. 
Die Abgrenzung von Planungsinformationen und Kontrolldaten stellt Daniel anhand von vier Charakteristika heraus:\footnote{Vgl. zu dieser Liste \cite{Daniel.1961}, S. 117, 119.}
\begin{itemize}\itemsep0pt
\item[1.]Erfassungsbereich -- Informationen zur \textit{Planung} sind, laut Daniel, nicht nach Funktionen gegliedert.\footnote{Vgl. zu diesem Listenpunkt \cite{Daniel.1961}, S. 117.} Tatsächlich sollen sie die Abgrenzung nach 
innerbetrieblichen Abteilungen überwinden und so eine Basis für ganzheitliche Pläne bieten. Konträr dazu wird sich bei der Erhebung 
von \textit{Kontrolldaten} stark an Unternehmensstrukturen orientiert, sodass diese Daten zur Leistungskontrolle von einzelnen Abteilungen genutzt werden können.
\item[2.]Betrachteter Zeitraum -- Informationen zur \textit{Planung} decken relativ lange Zeiträume ab und beschäftigen sich mit langfristigen Trends.\footnote{Vgl. zu diesem Listenpunkt \cite{Daniel.1961}, S. 117, 119.} 
Daten bezüglich der \textit{Kontrolle} hingegen werden regelmäßig erhoben und verarbeitet und betreffen somit ebenso kürzere Perioden.
\item[3.]Detailliertheitsgrad -- \textit{Planungsinformationen} sind auf wenig detaillierte Kurzdarstellungen von Sachverhalten beschränkt, während bei
\textit{Kontrolldaten} Präzision und minutiöse Darstellungen von Sitautionen eine große Rolle spielen. \footnote{Vgl. zu diesem und dem nächsten Listenpunkt \cite{Daniel.1961}, S. 119.}
\item[4.]Ausrichtung -- Informationen bezüglich der \textit{Planung} sollen Ausblicke auf die zukünftig zu erwartenden Enwtwicklungen bieten, während 
\textit{Kontrolldaten} vergangene Ergebnisse darstellen und Ursachen für diese aufzeigen. 
\end{itemize}
\noindent
Insgesamt nimmt Daniel seine Klassifizierung von \EF in drei Ebenen vor, wobei die beiden Kategorien Planung und Kontrolle die 
erste Ebene und gröbste Kategorisierung darstellen.\footnote{Vgl. zu diesem Satz \cite{Daniel.1961}, Grafik auf Seite S. 114.} 
Auf der nächsten Ebene unterteilt er den Abschnitt \textit{Planung} in drei Unterebenen:\footnote{Vgl. zu dieser Liste \cite{Daniel.1961}, S. 113-116.}
\begin{itemize}\itemsep0pt
\item[-]Umweltinformationen, sie beschreiben soziale, politische oder ökonomische Aspekte,
\item[-]Wettbewerbsinformationen, die die bisherige Wertentwicklung, Programme und Pläne der Wettbewerber erläutern,
\item[-]{interne Informationen, die auf Stärken und Schwächen des eigenen Unternehmens hinweisen.}
\end{itemize}
Diesen Unterkategorien ordnet er schließlich die eigentlichen \EF zu.
\\\noindent
Auf Seiten der \textit{Kontrolldaten} differenziert er zusätzlich nach Art der gelieferten Daten in zwei Unterkategorien:\footnote{Vgl. zu dieser Auflistung \cite{Daniel.1961}, S. 114.} 
\begin{itemize}\itemsep0pt
\item[-]finanzielle Daten,
\item[-]nicht-finanzielle Daten,
\end{itemize}
wobei sich innerhalb dieser Gruppen die Erfolgsfaktoren nominell entsprechen.\footnote{Vgl. \cite{Daniel.1961}, S. 114.} 
Für diese Konvergenz liefert Daniel keine explizite Erklärung, stellt sie aber als Besonderheit heraus.\footnote{Vgl. \cite{Daniel.1961}, S. 120.}
\\\noindent
\textit{Begründung:}\\\noindent
Daniel liefert keine gesonderte Begründung für seinen Ansatz. Er argumentiert jedoch, dass das Problem der hohen Anzahl von scheiternden 
Projekten in der Lücke zwischen statischen Informationssystemen und sich wandelnden Unternehmensstrukturen liegt.\footnote{Vgl. \cite{Daniel.1961}, S. 111.} 
Insbesondere weist er darauf hin, dass die bessere Handhabung von Daten zu einem Ersatz für das aufwendige Umstrukturieren von Positionen werden könnte.\footnote{Vgl. \cite{Daniel.1961}, S. 121.} 
Als Lösungsansatz hierfür schlägt er nachdrücklich die Implementierung oder Verbesserung eines entsprechenden Informationssystems vor, das den Aspekt der Planung 
ausdrücklich berücksichtigt.\footnote{Vgl. \cite{Daniel.1961}, S. 120.}