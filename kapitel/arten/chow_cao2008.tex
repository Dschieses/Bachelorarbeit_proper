\noindent
\subsubsection{Klassifizierung nach Chow und Cao (2008)}
\textit{Vorgehehen:}\\\noindent
Die Klassifizierung von Chow und Cao bezieht sich hauptsächlich auf agile Softwareprojekte.\footnote{Vgl. \cite{Chow.2008}, S. 961.}
In ihrer Studie zeigen sie fünf Dimensionen auf, in die sich die Faktoren einordnen lassen:\footnote{Vgl. zu dieser Liste \cite{Chow.2008}, S. 964.}
\begin{itemize}\itemsep0pt
\item[-]Unternnehmen,
\item[-]Menschen,
\item[-]Prozess,
\item[-]Technik,
\item[-]Projekt.
\end{itemize}
Für diese Faktorgruppe liefern sie insgesamt 12 Erfolgsfaktoren, die ihrer meinung nach die
\\\noindent
\textit{Begründung:}\\\noindent