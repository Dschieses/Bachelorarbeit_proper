\noindent
\subsubsection{Klassifizierung nach Hyväri (2006)}
\textit{Vorgehehen:}\\\noindent
Hyväri bietet eine Klassifizierung von EF in fünf Faktorgruppen:
\begin{itemize}\itemsep0pt
\item[-]Faktoren bezogen auf das Projekt.
\item[-]Faktoren bzgl. des Projektmanagements.
\item[-]Faktoren bzgl. Projektteams.
\item[-]Faktoren bzgl. des Unternehmens.
\item[-]Faktoren bzgl. der Rahmenbedingungen.
\end{itemize}
Diese Kategorisierung ist weitgehend kongruent zu dem von Belassi und Tukel gelieferten Schema. Hyväri geht in ihrer Interpretation
allerdings davon aus, dass auch Belassi und 
Tukel eine Unterteilung in fünf Kategorien vornehmen.\footnote{Vgl. \cite{Hyvari.2006}, S.33.}\\

%Hyväri geht jedoch nicht auf selbst definierte EF ein, sondern bezieht diese aus den Ergebnissen eines Literaturreviews 
%und kategorisiert sie anschließend in die genannten Faktorgruppen. 
%Größtenteils bezieht sich Hyväri bei diesem Literaturreview auf die Studie von Belassi und Tukel und sich selbst. 
%Diese Ef werden dann in einer Studie unter X Unternhmen/Teilnehmern einer Priorisierung unterzogen. 
%Das Vorgehen gliedert sich in eine allgemeine Befragung zur befragten Person und darauf folgend eine spezifische 
%Befragung der geeigneten Personen.
%In dem spezifischen Teil wird den Befragten die Liste der EF in den jeweiligen Kategorien vorgelegt. Aufgabe der 
%Interviewten war es dann, die drei – subjektiv - wichtigsten EF jeder Kategorie zu benennen. 
%Aus diesen Ergebnissen stellt Hyväri eine Liste der wichtigsten EF jeder Kategorie, abhängig von der Häufigkeit der 
%Benennung, auf. 
\textit{Begründung:}\\\noindent


