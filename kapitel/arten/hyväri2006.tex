\noindent
\subsubsection{Klassifizierung nach Hyväri (2006)}
\textit{Vorgehehen:}\\\noindent
Zweck der Studie von Hyväri war es, Erfolgsfaktoren des Projektmanagements 
auszuwerten.\footnote{Vgl. zu diesem und dem folgenden Satz \cite{Hyvari.2006}, S. 31.} Zusätzlich wollten sie prüfen, in welcher Beziehung die \EF und organisatorischen Variablen stehen.
Die hierzu aufgezeigte Klassifizierung von Erfolgsfaktoren untergliedert in fünf Faktorgruppen:\footnote{Vgl. zu dieser Liste \cite{Hyvari.2006}, S. 36.}
\begin{itemize}\itemsep0pt
\item[-]Faktoren bzgl. des Projekts,
\item[-]Faktoren bzgl. des Projektmanagements,
\item[-]Faktoren bzgl. Projektteams,
\item[-]Faktoren bzgl. des Unternehmens,
\item[-]Faktoren bzgl. der Rahmenbedingungen.
\end{itemize}
Diese Kategorisierung ist weitgehend kongruent zu dem von Belassi und Tukel\footnote{\cite{Belassi.1996}} gelieferten Schema. Hyväri unterteilt
jedoch die Gruppe \textit{Projektmanagement und Projektteam}, die von Belassi und Tukel als zusammenhängende Gruppe gesehen wird, in die zwei eigenständigen 
Klassen \textit{Projektmanagement} und \textit{Projektteam}.\footnote{Vgl. \cite{Hyvari.2006}, S. 33.}
Des Weiteren liefert Hyväri zu den einzelnen Gruppen jeweils drei Erfolgsfaktoren, die sie in ihrer Studie herausgearbeitet hat. Zu den projektbezogenen Faktoren 
zählen \textit{klare Ziele und Zielvorstellungen}, \textit{Engagement des Endbenutzers} und \textit{ausreichende Mittel und Ressourcen}.\footnote{Vgl. \cite{Hyvari.2006}, S. 34.}
Faktoren, welche sich auf das Projektmanagement beziehen, sind \textit{Engagement}, die \textit{Fähigkeit zur Kommunikation} sowie \textit{effektive Führung}.\footnote{Vgl. zu diesem und dem folgenden Satz \cite{Hyvari.2006}, S. 35.}
Zum Projektteam zählen die Faktoren wie \textit{Kommunikation}, \textit{Engagement} und \textit{technischer Hintergrund}.
Bezüglich des Unternehmens arbeitet Hyväri als wichtigsten Faktor die \textit{Unterstützung des Top-Managements} heraus.\footnote{Vgl. zu diesem und dem folgenden Satz \cite{Hyvari.2006}, S. 35-36.} Weitere Faktoren in dieser Gruppe sind \textit{klare Organisations- und Stellenbeschreibungen} sowie eine klare \textit{Projekt-Organisationsstruktur}.
Zur letzten Gruppe, den Rahmenbedingungen, zählt Hyvari den \textit{Kunden}, die \textit{technologische Umwelt} und die \textit{wirtschaftliche Umwelt}.\footnote{Vgl. \cite{Hyvari.2006}, S. 36.}
\\
\noindent\textit{Begründung:}\\\noindent
\todo{Begründung}
%Hyväri geht jedoch nicht auf selbst definierte EF ein, sondern bezieht diese aus den Ergebnissen eines Literaturreviews 
%und kategorisiert sie anschließend in die genannten Faktorgruppen. 
%Größtenteils bezieht sich Hyväri bei diesem Literaturreview auf die Studie von Belassi und Tukel und sich selbst. 
%Diese Ef werden dann in einer Studie unter X Unternhmen/Teilnehmern einer Priorisierung unterzogen. 
%Das Vorgehen gliedert sich in eine allgemeine Befragung zur befragten Person und darauf folgend eine spezifische 
%Befragung der geeigneten Personen.
%In dem spezifischen Teil wird den Befragten die Liste der EF in den jeweiligen Kategorien vorgelegt. Aufgabe der 
%Interviewten war es dann, die drei – subjektiv - wichtigsten EF jeder Kategorie zu benennen. 
%Aus diesen Ergebnissen stellt Hyväri eine Liste der wichtigsten EF jeder Kategorie, abhängig von der Häufigkeit der 
%Benennung, auf. 



