Im Folgenden ist der Aufbau der Bachelorarbeit aufgelistet. Neben der Kommentierung der einzelnen Kapitel ist hier auch der erwartete Umfang angegeben.\\
\\
\begin{longtable}{l p{17em}}
\textbf{1. Einleitung} & Die Einleitung umfasst Problemstellung, Zielsetzung, Vorgehensweise und Aufbau der Arbeit. (2-3 Seiten)\\
\midrule
\textbf{1.1 Problemstellung} & Mit welchem Problem befasst sich die Bachelorarbeit? Welche Relevanz haben diese Probleme und welche Antwort lässt sich finden? (\nicefrac{3}{4} Seite)\\
\midrule
\textbf{1.2 Zielsetzung} & Welches Ziel verfolgt die Bachelorarbeit? (\nicefrac{1}{4} Seite)\\
\midrule
\textbf{1.3 Vorgehensweise} & Wie wurde zur Zielerreichung der Arbeit vorgegangen? (\nicefrac{3}{4} Seite)\\
\midrule
\textbf{1.4 Aufbau der Arbeit} & Wie ist die Arbeit aufgebaut, sprich was ist im jeweiligen Kapitel der Arbeit zu finden? (\nicefrac{1}{2} Seite)\\
\toprule
\textbf{2.Grundlagen}  & Welche Grundlagen müssen geschaffen werden, damit ein Außenstehender diese Arbeit verstehen kann? (1-2 Seiten) \\
\midrule
\textbf{2.1 Definitionen} & Welche, nicht allgemein bekannten, Begriffe müssen zum Verständnis der Arbeit definiert werden? (1 Seite)\\
\toprule
\textbf{3. Analyse der Klassifizierungsarten} & Das Hauptkapitel befasst sich mit der Frage, welche Klassifizierungsarten in der Literatur adressiert werden. Welche Erklärungen für die jeweiligen Ansätze liefern die Autoren? (20-35 Seiten)\\
\toprule
\textbf{4.Fazit} & In welchem Umfang wurde das Ziel der Arbeit erreicht? Welches Fazit kann aus den Erkenntnissen der Arbeit gezogen werden? (1-2 Seiten) \\
\end{longtable}