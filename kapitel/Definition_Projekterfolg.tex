Projekterfolg ist das zusammenfassende Ergebnis der Beurteilung des Projekts hinsichtlich der Zielerreichung.\footnote{Vgl. \cite{DIN.200901} S.13.}
Neben den objektiv messbaren Zielkriterien, wie Ergebnis, Ter\-min- oder Budgettreue, hängt die Beurteilung des Projekterfolgs auch vom Standpunkt des jeweiligen Stakeholders ab.\footnote{Vgl. für diesen und den nächsten Satz \cite{Angermeier.o.J.}.}
So können auch Faktoren wie die Zufriedenheit des Auftraggebers oder die Bezahlung der Schlussrechnung als Kriterium für den Projekterfolg herangezogen werden. 