\documentclass[12pt,a4paper,oneside]{article}

%deutscher Zeichensatz
\usepackage[utf8]{inputenc}
\usepackage[T1]{fontenc}

%Brüche darstellen
\usepackage{nicefrac}

% deutsche Silbentrennung
\usepackage[ngerman]{babel}

%Farben
\usepackage{xcolor}
 
%Lange Tabellen
\usepackage{longtable}

%Lorem Ipsum
\usepackage{lipsum}

%Margins
\usepackage[top=2.5 cm, bottom=2.5 cm, left=5 cm, right=1 cm]{geometry}

% Sonderzeichen, wie Eurosymbol
\usepackage{textcomp}

%Format Überschriften
\usepackage{titlesec}
\titleformat{\section}{\bfseries}{\thesection.\quad}{12pt}{}
\titleformat{\subsection}{\bfseries}{\thesubsection\quad}{12pt}{}

%Linien in Tabellen
\usepackage{booktabs}

%Hyperlinks usw     
\usepackage[hyphens]{url}
\urlstyle{rm}


% Literaturverzeichnis
\usepackage[isbn=false,citestyle=authoryear,bibencoding=latin1,hyperref=true,bibstyle=authortitle,backend=bibtex,maxbibnames=30]{biblatex}
\usepackage[babel,german=guillemets]{csquotes}
\usepackage[colorlinks=false,hyperindex,pdfauthor={Peter Praeder},pdfborder={0 0 0}, pdfsubject={Bachelorarbeit},linkcolor={blue},citecolor={green},plainpages=false]{hyperref}
\addbibresource{bib/Literaturverzeichnis.bib}
\let\cite\textcite

%spezielles Abbildungsnummerierungszeugs
\usepackage{chngcntr}
\counterwithin{figure}{section}

%Abkürzungsverzeichnis
\usepackage[printonlyused]{acronym}    
\renewcommand{\bflabel}[1]{{#1\hfill}}

%Sprache Times New Roman
%\usepackage{mathptmx}
\usepackage{txfonts} %Schriftart Times New Roman
%Sprache: Deutsch
\selectlanguage{ngerman}

%Zeilenabstand
\usepackage{setspace}

% TOC, LOF, FIG Styles
\usepackage{tocloft, titletoc}  
\tocloftpagestyle{fancy}
\setlength{\cftaftertoctitleskip}{0em}
\renewcommand{\cftloftitlefont}{\bfseries}
%\renewcommand{\cftfigfont}{\bfseries}
\renewcommand{\cfttoctitlefont}{\bfseries}
\renewcommand{\cftlottitlefont}{\bfseries}
\titlecontents{section}     % set formatting for \section 
[2.3em]                     % adjust left margin
{\vspace{0.5em}}            % font formatting
{\hspace{-1.8em}.\contentslabel{0.7em}\hspace{1em}} % section label and offset
{\hspace*{-2.3em}}
{\titlerule*[1mm]{.}\contentspage}

\titlecontents{subsection}  % set formatting for \subsection 
[3em]                       % adjust left margin
{\vspace{0.5em}}            % font formatting
{\contentslabel{2.3em}}     % section label and offset
{\hspace*{-2.3em}}
{\titlerule*[1mm]{.}\contentspage}

\titlecontents{subsubsection}  % set formatting for \subsubsection 
[4.7em]                       % adjust left margin
{\vspace{0.5em}}            % font formatting
{\contentslabel{3.3em}}     % section label and offset
{\hspace*{-1.3em}}
{\titlerule*[1mm]{.}\contentspage}

\titlecontents{figure}      % set formatting for \figures 
[2.3em]                     % adjust left margin
{\vspace{0.5em}}            % font formatting
{\contentslabel{2.3em}}     % section label and offset
{\hspace*{-2.3em}}
{\titlerule*[1mm]{.}\contentspage}

\titlecontents{table}       % set formatting for \subsection 
[2.3em]                     % adjust left margin
{\vspace{0.5em}}            % font formatting
{\contentslabel{2.3em}}     % section label and offset
{\hspace*{-2.3em}}
{\titlerule*[1mm]{.}\contentspage}


%Kopfzeilen
\usepackage{fancyhdr} %Paket laden



%Fußzeilen
\usepackage[hang,flushmargin]{footmisc}    
\renewcommand*{\footnotelayout}{\footnotesize} % size of text
\renewcommand{\footnotemargin}{1.2em} % margin between text and number
\setlength{\footnotesep}{1.5em}    % space between footnotes
\setlength{\skip\footins}{2.5em}   % space between text & footnotes

%--------------Neue Kommandos-------------------------------------------
\newcommand{\KA}{Klassifizierungsart }
\newcommand{\KAS}{Klassifizierungsarten }
\newcommand{\EF}{Erfolgsfaktoren }
\newcommand{\ISP}{IS-Projekt }
\newcommand{\ISPS}{IS-Projekten }
\newcommand{\BA}{Bachelorarbeit }

\newcommand{\todo}[1]{\textbf{\textsc{\textcolor{red}{(TODO: #1)}}}}
\newcommand{\syn}[1]{\textbf{\textsc{\textcolor{red}{(SYNONYM: #1)}}}}

%--------------Worttrennung-------------------------------------------
\hyphenation{be-rück-sich-tigt IS--Pro-jekt pro-jekt-er-folg}

%--------------DOCUMENT-------------------------------------------------
\begin{document}

\pagestyle{plain}
\vspace*{1mm}

% Name
\thispagestyle{empty}
Peter Praeder

\vspace*{23mm}

% Bacheloararbeit
\begin{center}
\textbf{
Bacheloararbeit\linebreak
im Fach Allgemeine Wirtschaftsinformatik}
\end{center}

\vspace*{20mm}

% Titel
\begin{center}
\LARGE 
    Arten der Klassifizierung von Erfolgsfaktoren in IS-Projekten
\end{center}

\vspace*{8mm}

% Themensteller
\begin{center}
    Themensteller: Univ.-Prof. Dr. Werner Mellis
\end{center}

\vspace*{12mm}

% Vorgelegt
\begin{center}
    Vorgelegt in der Bachelorprüfung
\linebreak
    im Studiengang Wirtschaftsinformatik
\linebreak
    der Wirtschafts- und Sozialwissenschaftlichen Fakultät
\linebreak
    der Universität zu Köln
\end{center}
\vspace*{30mm}

% Köln, April 2013
\begin{center}
Köln, April 2013
\end{center}
\newpage


\pagenumbering{Roman}

\setcounter{page}{2}

\pagestyle{fancy}
\fancyhf{}
\renewcommand{\headrulewidth}{0.0pt} %obere Trennlinie
\fancyhead[C]{\nouppercase{\color{gray}{\thepage}}}
\tableofcontents{}

%--------------Abkürzungsverzeichnis-------------------------------------------------
\clearpage
\phantomsection
\addcontentsline{toc}{section}{Abkürzungsverzeichnis}
\section*{Abkürzungsverzeichnis}
\section*{Abkürzungsverzeichnis}
\renewcommand{\bflabel}[1]{{#1\hfill}}
\begin{acronym}[Kernbegriffe]
 \acro{EF}{Erfolgsfaktor}
 \acro{IS}{Informationssystem}
 \acro{IT}{Informationstechnologie}
\end{acronym}

%--------------Abbildungsverzeichnis-------------------------------------------------
\clearpage
\phantomsection
\addcontentsline{toc}{section}{Abbildungsverzeichnis}
\listoffigures

%--------------Tabellenverzeichnis-------------------------------------------------
\clearpage
\phantomsection
\addcontentsline{toc}{section}{Tabellenverzeichnis}
\listoftables
\clearpage
\onehalfspacing

%--------------Zeilennummerierung arabisch-------------------------------------------------
\pagenumbering{arabic}
%--------------Einleitung-------------------------------------------------
\section{Einleitung}
\subsection{Problemstellung}
Bereits seit mehreren Jahrzehnten werden die Faktoren, welche für den Erfolgs von IS-Pro\-jek\-ten verantwortlich sind, untersucht. Dabei wurden unterschiedliche Arten gefunden,
diese Erfolgsfaktoren in Gruppen einzuordnen, sprich zu klassifizieren.
Das ü\-ber\-ge\-ord\-ne\-te Forschungsproblem ergibt sich aus der, in der Fachliteratur fehlenden, umfassenden Ü\-ber\-sicht über diese Klassifizierungsarten, die die verschiedenen Autoren liefern.
Dieser Mangel erschwert die strukturierte Erfassung der maßgeblichen Komponenten, die das Gelingen eines IS-Projektes positiv beeinflussen und begünstigt somit die Wahrscheinlichkeit des Scheiterns dieser Projekte.
Der Mehrwert einer solchen Zusammenfassung besteht in der Möglichkeit, die Ergebnisse einer retrospektiven Analyse und die Gründe des Projekterfolgs leichter einzuordnen.
\\Zur Erstellung einer oben beschriebenen Übersicht muss man sich die folgende Frage stellen:
\textit{Welche unterschiedlichen Klassifizierungsarten von Erfolgsfaktoren von IS-Projekten werden in der Fachliteratur aufgezeigt und wie werden die Ansätze begründet?}\\
Hieraus definiert sich das eigentliche Forschungsproblem:
Die weitere Forschung im Bereich der Erfolgsfaktorenklassifizierung würde sich einfacher gestalten, wenn es eine Übersicht über die bereits vorherrschenden Klassifizierungsarten von Erfolgsfaktoren gäbe.
Ein systematisches Literraturreview kann dazu beitragen, die bereits bestehenden Klassifizierungen zu extrahieren und eventuell Defizite oder Forschungslücken aufzeigen.
Eine aus diesem Review resultierende Übersicht über die in der Fachliteratur aufgeführten Klassifizierungsarten von Erfolgsfaktoren in IS-Projekten würde zur Lösung des übergeordneten Forschungsproblems beitragen.
\subsection{Zielsetzung}
Das Hauptziel dieser Bachelorarbeit ist es, der Unübersichtlichkeit der Klassifizierungsarten von Erfolgsfaktoren in IS-Pojekten entgegenzuwirken.\\
Dazu ist zunächst zu klären, welche Arten in der Fachliteratur aufgezeigt werden. Zu\-sätz\-lich gilt es aufzuschlüsseln, welche Erklärungen die
jeweiligen Autoren für ihre Ansätze anbringen.
\subsection{Vorgehensweise}
Um einen Überblick über die in der Fachliteratur verwendeten Klassifikationen von Erfolgsfaktoren in IS-Projekten zu schaffen, wird ein 
systematisches Literaturreview durchgeführt. Dazu werden die Online-Portale von ACM digital library, AIS Electronic Library (AISeL), 
EBSCOhost("`Academic Search Complete” und "`Business Source Complete"), EmeraldInsight, IEEEXplore, ProQuest, ScienceDirect, SpringerLink und 
Wiley InterScience nach relevanten Texten durchsucht. Die Titel, Schlagwörter oder Zusammenfassungen der Literatur
sollen eine logische Verknüpfung der folgenden Begriffe enthalten: Erfolgsfaktor, Projekt, Software, Informationssystem und Informationstechnologie. 
Dabei muss darauf geachtet werden, dass sowohl die verschiedenen englischen Schreibweisen der Begriffe abgedeckt werden, als auch, dass „Success Factor“ und „Project“ und
mindestens einer der Begriffe „Software“, “Information System“ oder „Information Technology“ enthalten ist.\\ 
Die gefundenen Ergebnisse werden dann in einer Tabelle redundanzfrei festgehalten. Beim Lesen der Abstracts und gegebenenfalls der Texte wird festgestellt, ob diese sich tatsächlich auf den 
gewünschten Sachverhalt beziehen. Suchergebnisse, die in keinerlei inhaltlichem Zusammenhang zur Thematik stehen, werden hierbei verworfen.
Sollte in diesen Texten auf noch nicht berücksichtigte, relevante Literatur verwiesen werden, so wird diese nachgetragen.
Darauf folgt ein intensives Studium der relevanten Texte, in welchem die von den Autoren aufgezeigten Klassifizierungsarten und Begründungen extrahiert werden.



\subsection{Aufbau der Arbeit}
Im folgenden Kapitel werden zunächst elementare Grundlagen vermittelt, wie die Definition von Begriffen, die für das Verständnis dieser Arbeit wichtig sind.
Darauf folgt im dritten Kapitel der Schwerpunkt der Arbeit. Basierend auf dem Literraturreview werden die verschiedene Klassifizierungsarten dargestellt und zum Schluss zu einer 
Übersicht gruppiert.\todo{schöner}

blablabla autoren chronologisch, da sie gegebenenfalls aufeinander aufbauen
%--------------Grundlagen-------------------------------------------------
\section{Grundlagen}
\subsection{Begriffsdefinitionen}

\textbf{Informationssystem}\\\noindent
Ein Informationssystem (IS) ist ein System, welches in die Organisations-, Personal- und Technikstrukturen eines Unternehmens eingebunden ist.\footnote{Vgl. zu diesem Absatz \cite{Laudon.2009} S.17.} 
Es wird speziell für Zwecke eines bestimmten Unternehmens(teils) entwickelt und implementiert.
Zudem enthält es die dazu benötigte Anwendungssoftware und Daten.\\
\\\noindent
\textbf{Projekt}\\\noindent
Unter dem Begriff Projekt versteht man ein zeitlich definiertes Vorhaben, welches unternommen wird,
um eindeutige Produkte, Dienstleistungen oder Ergebnisse zu erstellen.\footnote{Vgl. \cite{ProjectManagementInstitute.2008} S.5.}
Ein Projekt ist im Wesentlichen durch die Einmaligkeit der Bedingungen in ihrer Gesamtheit gekennzeichnet, wie z.B. Zielvorgaben, zeitliche, finanzielle und personelle Begrenzungen.\footnote{Vgl. \cite{DIN.200901} S.11.}\\
\\\noindent
\textbf{Projekterfolg}\\\noindent
Projekterfolg ist das zusammenfassende Ergebnis der Beurteilung des Projekts hinsichtlich der Zielerreichung.\footnote{Vgl. \cite{DIN.200901} S.13.}
Neben den objektiv messbaren Zielkriterien, wie Ergebnis, Termintreue oder Budgettreue, hängt die Beutreilung des Projekterfogs auch von Standpunkt des jeweiligen Stakeholders ab.\footnote{Vgl. für diesen und den nächsten Satz \cite{Angermeier.o.J.}}
Zum Beispiel kann auch die Zufriedenheit des Auftraggebers oder die Bezahlung der Schlussrechnung als Kriterium für den Projekterfolg herangezogen werden. \\
\\\noindent
\textbf{Erfolgsfaktoren}\\\noindent
Erfolgsfaktoren im Alllgemeinen sind eine limitierte Anzahl von Faktoren, deren zufriedenstellendes Ergebnis erfolgreiche und 
wettbewerbsfähige Leistung für einzelne Bereiche oder das ganze Unternehmen sicherstellen.\footnote{Vgl. \cite{Bullen.1981} S.7.}
Bezogen auf Projekte sind dies Schlüsselfaktoren, die den Erfolg des Projektes fördern,\footnote{Vgl. \cite{Buschermohle.2010} zitiert nach \cite{Basten.2012}.} wie
z.B. Führungskompetenz und Erfahrung des Projektleiters, Kommunikation im Team oder Unterstützung des Managements.\\
Oft wird auch von kritischen Erfolgsfaktoren gesprochen, wobei es für diesen Begriff keine allgemein anerkannte Definition gibt.\footnote{Vlg. \cite{Basten.2012}, S. 59.}\\
\\\noindent
\textbf{Klassifizierung}\\\noindent
Klassifizierung beschreibt den Prozess der systematischen Zuweisung ähnlicher Objekte zu Objektklassen.\footnote{Vgl. \cite{Elmasri.2009}, S. 118.}
Bezogen auf \EF ist eine \KA demnach die Einordnung von \EF in Kategorien.


\subsection{Klassifizierung von \EF}
Rockart\footnote{\cite{Rockart.1979}} entwickelte als einer der Ersten den Ansatz, Erfolgsfaktoren zu identifizieren und Unternehmensleistung zu messen.\footnote{Vgl. für diesen und den folgenden Satz \cite{Chow.2008}, S. 962.}
Bullen und Rockart\footnote{\cite{Bullen.1981}} sowie Rockart und Crescenzi\footnote{\cite{Rockart.1984}} etabliereten und verfeinerten den Ansatz daraufhin.\\
Seit Beginn der Forschung zu Erfolgsfaktoren von Projekten, wurden immer wieder Auflistungen von Faktoren 
geliefert, jedoch wurde nur eine geringe Priorität auf das Klassifizieren von EF gelegt.\\\noindent
Der Vorteil einer solchen Einordnung in Gruppen liegt darin, dass man, anstatt einzelne Erfolgsfaktoren zu analysieren, zunächst die Gruppen identifizieren kann, in die die Faktoren
einzuordnen sind.\footnote{Vgl. zu diesem und dem folgenden Satz \cite{Belassi.1996}, S. 142.} Daraufhin können dann die kombinierten Auswirkungen der Faktoren auf den Projekterfolg zu bestimmen.

Die Klassifizierung von EF kann jedoch bei der Analyse der Wechselwirkung zwischen diesen und deren Konsequenzen helfen.\footnote{Vgl. \cite{Belassi.1996}, S. 142.} 



%\textit{Motivation:}\\\noindent
%Die Motivation und das formulierte Problem aus Sicht der Autoren liegt in der Ambiguität der Beurteilung von Projekterfolg.S141-142
%Für diese sehen die Autoren zwei Gründe: Der erste und untergeordnete Grund basiert auf der subjektiven Einschätzung des 
%Projektergebnisses. Auch Projekte, die das Management als gescheitert ansieht, können für den Kunden ein Erfolg sein (ZITAT)141.
%Der primäre Problemansatz liegt aber in der starken Variation der Auflistungen von EF in der Fachliteratur. Dies begünstigt (??) den 
%generellen Trend zur Einordnung von EF in Tabellen, anstatt diese strukturiert bestimmten Kriterien zuzuordnen und so zu kategorisieren. 
%Weitgehend sind es oftmals nicht nur die einzelnen EF, die einen großen Einfluss auf den Erfolg eines Projektes haben, sondern auch das Zusammenspiel 
%von mehreren Faktoren aus unterschiedlichen Kategorien und in unterschiedlichen Phasen des Projektes.\\

%--------------Hauptteil-------------------------------------------------
\clearpage
\section{Klassifizierungsarten in der Literatur}
Im Folgenden werden zunächst die verschiedenen Arten, Erfolgsfaktoren zu klassifizieren mitsamt der Begründung der Autoren dargestellt.
Auf dieser Basis wird anschließend eine Übersicht erstellt und eine Schnittmengenbetrachtung vorgenommen. 
\subsection{Klassifizierungsarten und deren Begründungen}
%--------------Arten-------------------------------------------------
\noindent
\subsubsection{Klassifizierung nach Daniel (1961)}
\textit{Vorgehehen:}\\\noindent
Daniel erwähnt als einer der Ersten überhaupt den Begriff Erfolgsfaktor generell, aber auch im Zusammenhang mit Informationssystemen.\footnote{Vgl. \cite{Daniel.1961}, S. 116 sowie \cite{Fortune.2006}, S. 53 und \cite{Rockart.1979}, S. 85.}
Ihm zufolge gibt es in den meisten Branchen drei bis sechs Faktoren, denen besondere Aufmerksamkeit geschenkt werden muss,
um eine Unternehmung positiv zu beeinflussen, da sie den Erfolg maßgeblich bedingen.\footnote{Vgl. \cite{Daniel.1961}, S. 116.}
Prinzipiell beabsichtigt Daniel mit seinem Artikel in erster Linie, die Relevanz eines eigenständigen Informationssystems 
herauszustellen.\footnote{Vgl. zu diesem und dem folgenden Satz \cite{Daniel.1961}, S. 113.}
Seiner Ansicht nach liegt der Schlüssel zu einem dynamischen und nützlichen System in zwei entscheidenden Elementen des Managementprozesses:
\begin{itemize}\itemsep0pt
\item[-]Planung und
\item[-]Kontrolle,\footnote{Vgl. zu dieser Liste \cite{Daniel.1961}, S. 113, 114, 117.}
\end{itemize}
wobei Daniel primär auf den Aspekt der \textit{Planung} fokussiert.\footnote{Vgl. zu diesem Teilsatz \cite{Daniel.1961}, S. 113.} 
\textit{Planung} definiert er im Folgenden als Zielsetzung, Strategieformulierung und Entscheidungsfindung zwischen alternativen
Anlagen oder Handlungsoptionen,\footnote{Vgl. zu diesem Teilsatz \cite{Daniel.1961}, S. 113.} während er \textit{Kontrolle} nicht gesondert definiert. 
Die Abgrenzung von Planungsinformationen und Kontrolldaten stellt Daniel anhand von vier Charakteristika heraus:\footnote{Vgl. zu dieser Liste \cite{Daniel.1961}, S. 117, 119.}
\begin{itemize}\itemsep0pt
\item[1.]Erfassungsbereich -- Informationen zur \textit{Planung} sind, laut Daniel, nicht nach Funktionen gegliedert.\footnote{Vgl. zu diesem Listenpunkt \cite{Daniel.1961}, S. 117.} Tatsächlich sollen sie die Abgrenzung nach 
innerbetrieblichen Abteilungen überwinden und so eine Basis für ganzheitliche Pläne bieten. Konträr dazu wird sich bei der Erhebung 
von \textit{Kontrolldaten} stark an Unternehmensstrukturen orientiert, sodass diese Daten zur Leistungskontrolle von einzelnen Abteilungen genutzt werden können.
\item[2.]Betrachteter Zeitraum -- Informationen zur \textit{Planung} decken relativ lange Zeiträume ab und beschäftigen sich mit langfristigen Trends.\footnote{Vgl. zu diesem Listenpunkt \cite{Daniel.1961}, S. 117, 119.} 
Daten bezüglich der \textit{Kontrolle} hingegen werden regelmäßig erhoben und verarbeitet und betreffen somit ebenso kürzere Perioden.
\item[3.]Detailliertheitsgrad -- \textit{Planungsinformationen} sind auf wenig detaillierte Kurzdarstellungen von Sachverhalten beschränkt, während bei
\textit{Kontrolldaten} Präzision und minutiöse Darstellungen von Sitautionen eine große Rolle spielen. \footnote{Vgl. zu diesem und dem nächsten Listenpunkt \cite{Daniel.1961}, S. 119.}
\item[4.]Ausrichtung -- Informationen bezüglich der \textit{Planung} sollen Ausblicke auf die zukünftig zu erwartenden Enwtwicklungen bieten, während 
\textit{Kontrolldaten} vergangene Ergebnisse darstellen und Ursachen für diese aufzeigen. 
\end{itemize}
\noindent
Insgesamt nimmt Daniel seine Klassifizierung von \EF in drei Ebenen vor, wobei die beiden Kategorien Planung und Kontrolle die 
erste Ebene und gröbste Kategorisierung darstellen.\footnote{Vgl. zu diesem Satz \cite{Daniel.1961}, Grafik auf Seite S. 114.} 
Auf der nächsten Ebene unterteilt er den Abschnitt \textit{Planung} in drei Unterebenen:\footnote{Vgl. zu dieser Liste \cite{Daniel.1961}, S. 113-116.}
\begin{itemize}\itemsep0pt
\item[-]Umweltinformationen, sie beschreiben soziale, politische oder ökonomische Aspekte,
\item[-]Wettbewerbsinformationen, die die bisherige Wertentwicklung, Programme und Pläne der Wettbewerber erläutern,
\item[-]{interne Informationen, die auf Stärken und Schwächen des eigenen Unternehmens hinweisen.}
\end{itemize}
Diesen Unterkategorien ordnet er schließlich die eigentlichen \EF zu.
\\\noindent
Auf Seiten der \textit{Kontrolldaten} differenziert er zusätzlich nach Art der gelieferten Daten in zwei Unterkategorien:\footnote{Vgl. zu dieser Auflistung \cite{Daniel.1961}, S. 114.} 
\begin{itemize}\itemsep0pt
\item[-]finanzielle Daten,
\item[-]nicht-finanzielle Daten,
\end{itemize}
wobei sich innerhalb dieser Gruppen die Erfolgsfaktoren nominell entsprechen.\footnote{Vgl. \cite{Daniel.1961}, S. 114.} 
Für diese Konvergenz liefert Daniel keine explizite Erklärung, stellt sie aber als Besonderheit heraus.\footnote{Vgl. \cite{Daniel.1961}, S. 120.}
\\\noindent
\textit{Begründung:}\\\noindent
Daniel liefert keine gesonderte Begründung für seinen Ansatz. Er argumentiert jedoch, dass das Problem der hohen Anzahl von scheiternden 
Projekten in der Lücke zwischen statischen Informationssystemen und sich wandelnden Unternehmensstrukturen liegt.\footnote{Vgl. \cite{Daniel.1961}, S. 111.} 
Insbesondere weist er darauf hin, dass die bessere Handhabung von Daten zu einem Ersatz für das aufwendige Umstrukturieren von Positionen werden könnte.\footnote{Vgl. \cite{Daniel.1961}, S. 121.} 
Als Lösungsansatz hierfür schlägt er nachdrücklich die Implementierung oder Verbesserung eines entsprechenden Informationssystems vor, das den Aspekt der Planung 
ausdrücklich berücksichtigt.\footnote{Vgl. \cite{Daniel.1961}, S. 120.}

\noindent
\subsubsection{Klassifizierung nach Rockart (1979)}
\textit{Vorgehehen:}\\\noindent
In seiner Studie unterteilt Rockart Erfolgsfaktoren in folgende Gruppen:\footnote{Vgl. zu dieser Liste \cite{Rockart.1979}, S. 92.}
\begin{itemize}\itemsep0pt
\item[-]Faktoren bezüglich der Kontrolle,
\item[-]Faktoren bezüglich der Erstellung.
\end{itemize}
\textit{Begründung:}\\\noindent
Rockart liefert keine explizite Begründung für seinen Ansatz, geht jedoch in seinem gesamten Artikel auf die
generelle Relevanz von Erfolgsfaktoren ein. 
Hierzu führt er an, dass z.B. die Identifikation von Erfolgsfaktoren eine klare Definition
über den Umfang der zu beschaffenden Informationen ermöglicht und eine teure Mehraquise beschränkt.\footnote{Vgl. \cite{Rockart.1979}, S. 87.}


\noindent
\subsubsection{Klassifizierung nach Slevin und Pinto (1987)}
\textit{Vorgehen:}\\\noindent
Slevin und Pinto beziehen sich in ihrer Studie auf Erfolgsfaktoren im allgemeinen Projektmanagement.
Sie erstellen eine Liste von zehn Erfolgsfaktoren für die erfolgreiche Implementierung
von Projekten, welche sie zunächst zwei verschiedenen Projektphasen zuordnen.\footnote{Vgl. zu diesem und dem folgenden Satz \cite{Slevin.1987}, S. 3.} 
Die Faktoren \textit{Projektrichtung}, \textit{Top-Management-Unterstützung} und \textit{Zeitplan} teilen sie in die Planungsphase eines Projektes ein. 
Die restlichen sieben \EF,wie zum Beispiel Kundenkonsultation, technische Aufgaben oder Akzeptanz des 
Kunden,\footnote{Vgl. zu den Erfolgsfaktoren \cite{Slevin.1987}, S. 2.} ordnen die Autoren der Implementierungs - 
bzw. Aktionsphase zu.\\\noindent
Alle \EF klassifizieren sie schließlich in:\footnote{Vgl. zu dieser Liste \cite{Slevin.1987}, S. 3.}
\begin{itemize}\itemsep0pt
\item[-]Strategische Faktoren,
\item[-]Taktische Faktoren.
\end{itemize}
Die strategischen Faktoren beziehen sich dabei auf das Aufstellen von übergeordneten Zielen und die Planung, wie diese zu erreichen sind.\footnote{Vgl. zu diesem und dem folgenden Satz \cite{Slevin.1987}, S. 3.}
Zu den taktischen Faktoren zählen, laut Slevin und Pinto, die menschlichen, technischen und finanziellen Ressourcen, die zur Erreichen der
strategischen Ziele notwendig sind. Aufbauend auf diesem Konzept entwickeln Slevin und Pinto eine Agenda mit fünf Schritten, die, bezüglich
der Klassen Strategie und Taktik, die Wahrscheinlichkeit einer erfolgreichen Implementierung erhöhen:\footnote{Vgl. zu diesem Satz und der Liste \cite{Slevin.1987}, S. 8-9.}
\begin{itemize}\itemsep0pt
\item[1.]Nutzen eines Multi-Faktor-Modells
\item[2.]Frühes strategisches Denken im Projektlebenszyklus
\item[3.]Im fortschreitenden Lebenszyklus mehr taktisches Denken
\item[4.]Bewusste Planung und Kommunikation des Übergangs von Strategie zu Taktik
\item[5.]Fokus gleichermaßen auf Strategie und Taktik
\end{itemize}
\noindent\textit{Begründung:}\\\noindent
Laut Slevin und Pinto setzt ihr Modell am ersten Schritt der Agenda an, da
es für Projektmanager essentiell sei, mehrere Faktoren gleichzeitig zu berücksichtigen.\footnote{Vgl. zu diesem Satz und dem Rest des Absatzes \cite{Slevin.1987}, S. 8.}
Ihr Studie bietet zehn Erfolgsfaktoren in einem Rahmenkonzept zur Projektimplementierung.
Innerhalb dieses Rahmenkonzeptes werden verschiedenen Faktoren, abhängig von Phase des Projektes, unterschiedliche Relevanz zugeordnet.
Die Priorisierung der \EF erfolgt also sowohl abhängig von zeitlichem Status des Projektes, als auch von der Kategorie des Faktors.



\noindent
\subsubsection{Klassifizierung nach Belassi und Tukel (1996)}
\textit{Vorgehen:}\\\noindent
Belassi und Tukel beziehen sich in ihrer Studie von 1996 auf Projektmanagement im Allgemeinen und nicht spezifisch auf
Erfolgsfaktoren von IS-Projekten.\footnote{Vgl. \cite{vanScoter.2011}, S. 4.}
Das Schema, welches Belassi und Tukel zur Klassifizierung von Erfolgsfaktoren anwenden, beinhaltet vier Kategorien:\footnote{Vgl. zu dieser Liste \cite{Belassi.1996}, S. 143-145.}
\begin{itemize}\itemsep0pt
\item[-]Faktoren bezogen auf die Eigenschaften des Projektes, wie Größe, Dauer und Projektnetzwerk,
\item[-]Faktoren bzgl. des Projektmanagements und der Teammitgliedern, in ihrer Wichtigkeit abhängig von der Phase des Projektes,
\item[-]Faktoren bzgl. der Organisation/des Unternehmens, hier ist als wichtigster Faktor die Unterstützung durch das Top-Management zu erwähnen,\footnote{Vgl. zum zweiten Halbsatz \cite{Belassi.1996}, S. 146.}
\item[-]Faktoren die außerhalb des Unternehmens liegen, aber dennoch Einfluss auf das Projekt haben wie zum Beispiel politische, ökonomische oder soziale Faktoren.
\end{itemize}
\textit{Begründung:}\\\noindent
Ziel von Belassi und Tukel war es, mit ihrer Studie eine Möglichkeit aufzuzeigen, Erfolgsfaktoren zu klassifizieren und deren 
Einfluss auf die Projektperformance zu identifizieren.\footnote{Vgl. \cite{Belassi.1996}, S. 142.}
Sie begründen, dass der Vorteil des hier aufgezeigten Schemas in der Möglichkeit liegt, die Abhängigkeit zwischen Projekterfolg
und den Faktorgruppen Projektmanager und/oder Projekt und/oder externe Faktoren, leichter aufzuschlüsseln.\footnote{Vgl. zu diesem Satz und dem Rest des Absatzes \cite{Belassi.1996}, S. 143-144.}
Das so entwickelte Rahmenkonzept fördert das Verständnis des Projektmanagers für die Beziehungen der Faktoren zwischen den Gruppen. 
Dies führt dazu, dass der Projektmanager sein Projekt präziser beurteilen und überwachen kann. Zudem ist dieses Konzept sehr leicht an spezifische Projekte anzupassen und kann so einfach erweitert werden.


\noindent
\subsubsection{Klassifizierung nach Reel (1999)}
\textit{Vorgehen:}\\\noindent
Reel liefert in seiner Studie, welche sich ausschließlich auf Softwareprojekte bezieht, keine Klassifizierung im eigentlichen Sinne.
Vielmehr listet er fünf Faktoren auf, die seine Meinung nach für das Management eines erfolgreichen Softwareprojekts essentiell sind.
\begin{itemize}
  \item[-]{"`Auf dem richtigen Fuß beginnen"',}
  \item[-]{Erhaltung der Dynamik,}
  \item[-]{den Fortschritt verfolgen,}
  \item[-]{kleine Entscheidungen treffen,}
  \item[-]{einrichten von post-mortem Analsysen.}  
\end{itemize}
\textit{Begründung:}\\\noindent

\subsubsection{Klassifizierung nach Milis und Mercken (2002)}
\textit{Vorgehen:}\\\noindent
In ihrer Studie bezüglich der Einführung von Informations- und Kommunikationstechnologie in belgischen Banken und Versicherungen, 
fanden Milis und Mercken eine große Anzahl von Erfolgsfaktoren. Zur Strukturierung dieser Faktoren entwickelten sie folgendes Rahmenkonzept:
Sie unterteilen grundsätzlich in vier Kategorien:
\begin{itemize}\itemsep0pt
  \item[-]{Faktoren, die Einfluss auf die Zielkongruenz nehmen.}
  \item[-]{Faktoren, die sich auf das Projektteam beziehen.}
  \item[-]{Faktoren, die die Akzeptanz des Projekts und dessen Ergebnis beeinflussen.}
  \item[-]{Faktoren, die sich auf das Vorgehen bei der Implementierung beziehen.}  
\end{itemize}
\textit{Begründung:}\\\noindent
Zusätzlich zu diesem System unterteilung in 8 Kategorien (Tabelle 1)


\noindent
\subsubsection{Klassifizierung nach Kendra und Taplin (2004)}
\textit{Vorgehen:}\\\noindent
//Fokussiert auf Projektmamagement. Laut Kendra und Taplin müssen Organisationen, um bei der Einführung von neuen Mangementpraktiken erfolgreich zu sein, eine gemeinsame Basis an Werten und Ansichten (Projektmanagementkultur) entwickeln. Der Projektmanagementerfolg begründet sich auf den 4 Dimensionen von Projekterfolg: Fähigkeiten und Kompetenzen  des Projektmanagements (Mikrosozial), Organisatorische Struktur auf Projektebene (Makrosozial), Leistungsmessungssysteme (Mikrotechnisch), Unterstützung durch das Management (Makrotechnisch).
Begründung:
Die von Kendra und Taplin gelieferte Klassifizierung bietet ein Rahmenkonzept, mit der Unternehmen eine Bewertung ihres aktuellen Managementpotentials durchführen können. Hierbei werden die Fähigkeiten des Projektmanagements, die Hauptgeschäftsprozesse, der projektspezifische Ressourcenverbrauch und die nötigen weiteren Schritte, um die Verbesserung des Projektmanagements  zu unterstützen, herausgestellt. Darauf aufbauend kann eine Veränderung in ineffizienten Projektbereichen initiiert werden, sodass die Projektperformance gesteigert wird.

Bezieht EF von anderen Autoren
Interview von IT-leadern 
Unterschieden wird nach zwei Hauptkategorien: Sozial und Technisch. Eine spezifischere Unterteilung in Makrosoziale/-technische und Mikrosoziale/-technische Faktoren liefert die endgültige Klassifizierung.
Kategorisierung:
\begin{itemize}
\item[-]Mikrosozial: Fähigkeiten und Kompetenzen  des Projektmanagements 
\item[-]Makrosozial: Organisatorische Struktur auf Projektebene
\item[-]Mikrotechnisch: stellt ein Leistungsmessungssystem dar - individuelle Maßstäbe, um Performance im organisatorischen und projektbezogenen Bereich zu überwachen
\item[-]Makrotechnisch: Unterstützung durch das Management, Bilden von strukturierten Geschäftsprozessen und -rahmenkonzepten 
\end{itemize}
\textit{Begründung:}\\\noindent
Wobei EF wie Team-Performance und Kundenzufriedenheit in den Mikrotechnischen Bereich fallen

Scio-technical system theory (taylor und felton 1993)
Open system theory (emery und Purser 1996)


\noindent
\subsubsection{Klassifizierung nach Salmeron und Herrero (2005)}
\textit{Vorgehehen:}\\\noindent
In ihrer Studie beziehen sich Salmeron und Herrero auf die Erfolgsfaktoren %bei der Implementierung
von \textit{Executive Information Systems}, also 
Führungsinformationssystemen (\acs{FIS}) oder Management-Unterstützungs-Systemen (\acs{MUS}).
\footnote{Vgl. zu diesem und dem Folgenden Satz \cite{Salmeron.2005}, S. 1.}
Unter Nutzung des Analytischen Hierarchie Prozesses (\acs{AHP}) liefern sie eine Priorisierung der Erfolgsfaktoren, die sie als kritische \ac{EF} definieren.
Der \ac{AHP} ist eine starke und flexible Methode, die von dem Mathematiker Thomas Saaty entwickelt wurde,\footnote{Zur Entwicklung des AHP siehe \cite{Saaty.1977} sowie \cite{Saaty.1980}} 
um Entscheidungsprozesse zu unterstützen und Prioritäten zwischen verschiedenen Attributen zu setzen.\footnote{Vgl. zu diesem Satz und dem Rest des Absatzes \cite{Salmeron.2005}, S. 3.}
Dazu muss in einem ersten Schritt das Entscheidungsproblem in eine Hierarchie von zusammenhängenden Elementen aufgeschlüsselt werden.
Darauf folgen im zweiten Schritt paarweise Vergleiche der Elemente und Berechnungen der Attributgewichtungen.
In einem letzten Schritt müssen die Gewichtungen der Kategorien berechnet werden.\\
Die Klassifizierung, die Salmeron und Herrero aus der Verwendung des \ac{AHP} ableiten, 
unterteilt die Erfolgsfaktoren in folgende Gruppen:\footnote{Vgl. zu dieser Liste \cite{Salmeron.2005}, S. 4.}
\begin{itemize}\itemsep0pt
\item[-]Personelle Ressourcen,
\item[-]Informationen und Technologie,
\item[-]System Wechelwirkungen.
\end{itemize}
\noindent
Zur Faktorgruppe \textit{Personelle Ressorcen} zählen Sie die Faktoren \textit{Benutzereinbeziehung},die \textit{Existenz von Unterstützung des Auftraggebers}
sowie die \textit{Notwendigkeit von kompetentem und ausgewogenem Personal}.\footnote{Vgl. \cite{Salmeron.2005}, S. 4.}
Die Kategorie \textit{Informationen und Technologie} beinhaltet, laut Salmeron und Herrero, \textit{geeignete Hardware und Software} sowie \textit{richtige Informationen}.\footnote{Vgl. zu diesem und dem folgenden Satz \cite{Salmeron.2005}, S. 5.}
Zur letzten Gruppe, den \textit{System Wechselwirkungen} zählen sie die Faktoren \textit{schnelle Prototypentwicklung}, \textit{maßgeschneiderte Systemlösungen} sowie ein \textit{flexibles und sensitives System}.
\\\noindent\textit{Begründung:}\\\noindent

\input{kapitel/arten/hyväri2006.tex}

\noindent
\subsubsection{Klassifizierung nach Fortune und White (2006)}
\textit{Vorgehen:}\\\noindent
Einen weiteren Ansatz zur Klassifizierung liefern Fortune und White. Sie beschreiben in ihrer Studie die Verwendung des 
\ac{FSM},\footnote{Zur Beschreibung des FSM siehe \cite{White.2009}.} welches ursprünglich von Bignell und Fortune entwickelt wurde.\footnote{Vgl. \cite{Fortune.2006}, S. .}
%Das formale System im Kern des Modells umfasst ein Subsystem zur Entscheidungsfindung, ein Subsystem zur Leistungsüberwachung
%sowie eine Reihe von Subsystemen und Elementen die die Aufgaben des Systems und folglich seine Umwandlungen von Inputs in
%Outputs durchführen. 
Basierend auf den Komponenten des \ac{FSM} kommen Fortune und White zu folgender Klassifizierung:\footnote{Vgl. zu dieser Liste \cite{Fortune.2006}, S. 57.}
\begin{itemize}\itemsep0pt
\item[-]{Ziele und Vorgaben}
\item[-]{Leistungsüberwachung}
\item[-]{Entscheidungsträger}
\item[-]{Transformationen}
\item[-]{Kommunikation}
\item[-]{Umgebung}
\item[-]{Beschränkungen}
\item[-]{Ressourcen}
\item[-]{Kontinuität}
\end{itemize}
\textit{Begründung:}\\\noindent
Als Begründung zeigen Fortune und White zunächst zwei Kritikpunkte an dem bestehenden Ansatz von Erfolgsfaktoren auf, denen ihre Klassifizierung entgegen wirken soll.\\\noindent
Sie bemängeln als Erstes, dass bei der Auflistung von einzelnen Faktoren nicht auf die Beziehung zwischen diesen eingegangen wird,
dies sei jedoch so wichtig, wie die Faktoren selbst.\footnote{Vgl. \cite{Fortune.2006}, S. 54.}
Gegen diesen Punkt führen sie an, dass das \ac{FSM} genauso stark auf die Beziehung zwischen den Faktoren, wie auf die Faktoren selbst fokussiert.
Als zweiten Punkt kritisieren sie nach Larsen und Myers, dass der Ansatz der Faktoren dazu neigt, die Implementation als
statischen Prozess, anstelle eines dynamischen Phänomens zu sehen.\footnote{Vgl. \cite{Larsen.1999}, S. 398 sowie \cite{Fortune.2006}, S. 54.}
Bezüglich dieses Kritikpunktes argumentieren sie, dass das \ac{FSM} ein dynamisches Modell eines Systems ist, welches auf Entscheidungsfindung reagiert und mit seiner
Umwelt interagiert.\footnote{Vgl. \cite{Fortune.2006}, S. 57.}


\noindent
\subsubsection{Klassifizierung nach Chow und Cao (2008)}
\textit{Vorgehehen:}\\\noindent
Chow und Chao behandeln in ihrer Studie die Erfolgsfaktoren von agilen Softwareprojekten.\footnote{Vgl. \cite{Chow.2008}, S. 961.}
In ihrer Studie zeigen sie fünf Dimensionen auf, in die sich die Faktoren einordnen lassen:\footnote{Vgl. zu dieser Liste \cite{Chow.2008}, S. 964.}
\begin{itemize}\itemsep0pt
\item[-]Unternnehmen,
\item[-]Menschen,
\item[-]Prozess,
\item[-]Technik,
\item[-]Projekt.
\end{itemize}
Für diese Faktorgruppe liefern sie insgesamt 12 Erfolgsfaktoren, die ihrer Meinung nach die vier Erfolgsdimensionen der agilen Softwareentwicklung \textit{Qualität}, \textit{Zeit}, \textit{Kosten} und \textit{Umfang} beeinflussen.\footnote{Vgl. \cite{Chow.2008}, S. 964.}
\\\noindent
\textit{Begründung:}\\\noindent

\noindent
\subsubsection{Klassifizierung nach Pankratz und Loebbecke (2011)}
\textit{Vorgehehen:}\\\noindent

\footnote{Vgl. zu dieser Liste \cite{Pankratz.2011}, S. 7.}
\begin{itemize}\itemsep0pt
\item[-]{Effizienz der Projektleistung,}
\item[-]{Beziehung zwischen Kunde und Auftragnehmer,}
\item[-]{Gewährleistung der Produktqualität,}
\item[-]{Projektsicherheit,}
\item[-]{Beziehung zwischen Management und Projekt,}
\item[-]{Motivation der Teammitglieder,}
\item[-]{Qualifikation der Teammitglieder,}
\item[-]{Zusammensetzung des Teams,}
\item[-]{Verantwortung der Teammitglieder,}
\item[-]{Konzentration der Teammitglieder auf das Projekt,}
\item[-]{Verwaltung von Erwartungen der Teammitglieder,}
\item[-]{klare Zielvorgaben,}
\item[-]{Transparenz im Projekt,}
\item[-]{Kommunikation im Projekt,}
\item[-]{Planung, Überwachung und Kontrollen,}
\item[-]{systematischer Ansatz,}
\item[-]{Charakteristik des Projektmanagers,}
\item[-]{Generelle Bedingungen,}
\item[-]{Sonstiges.}
\end{itemize}
\textit{Begründung:}\\\noindent

\noindent
\subsubsection{Klassifizierung nach Sudhakar (2012)}
\textit{Vorgehehen:}\\\noindent
Sudhakar stellt in seiner Studie folgende Kategorien vor, in die sich die Faktoren einordnen lassen, die er bei einem gründlichen
Literaturreview gefunden hat:\footnote{Vgl. zu dieser Liste \cite{Sudhakar.2012}, S. 545.}
\begin{itemize}\itemsep0pt
\item[-]Kommunikationsfaktoren,
\item[-]Technische Faktoren,
\item[-]Unternehmensfaktoren,
\item[-]{Umweltfaktoren,}
\item[-]{Produktfaktoren,}
\item[-]{Teamfaktoren,}
\item[-]{Projektmanagementfaktoren.}
\end{itemize}
\textit{Begründung:}\\\noindent
Sudhakar führt an, dass sich seine Forschung sowohl ergänzend auf die Literatur, als auch auf die Praxis auswirkt.\footnote{Vgl. zu diesem Absatz \cite{Sudhakar.2012}, S. 553.}
Auf Seiten der Literatur wirkt seine Forschung dahingehend, dass sie dem Literaturbestand den wichtigen Aspekt des 
konzeptionellen Modells von Erfolgsfaktoren verschiedener Dimensionen hinzufügt.
In der Praxis können Projektmanager von den gefundenen Erfolgsfaktoren und deren Kategorisierung profitieren.
Somit könne weiterhin auf das höchste Ziel des Projektmanagements,den Erfolg des Projekts, hingearbeitet werden.

%Sudhakar begründet seine Klassifizierung zunächst mit der Feststellung, dass dieser Ansatz auf Projekte in anderen Branchen und 
%Ländern adaptiert werden kann.\footnote{Vgl. zu diesem Absatz \cite{Sudhakar.2012}, S. 554.}
%Darüber hinaus kann, laut Sudhakar, sein Ansatz auf multinationale Projekte, die mehrere Branchen oder Unternehmen involvieren, 
%ausgeweitet werden.


%Dies führte global gesehen zu einer gesteigerten Erfolgsrate von Projekten.
%--------------Arten Ende-------------------------------------------------
\subsection{Übersicht}

\subsection{Schnittmengenbetrachtung}
\subsubsection{Konstante \EF}
\subsubsection{Klassifizierung der konstanten \EF}

%--------------Fazit-------------------------------------------------
\section{Fazit}
Aufgrund von einer relativ kurzen Bearbeitungszeit von nur 7 Wochen mussten einige beschränkungen zugelassen werden.
Zum Beispiel musste die ursprüngliche Anzahl von XX Suchdatenbanken nach einer überwältigenden Trefferzahl von über 500
reduziert werden. Dabei fielen Datenbanken heraus wie XXX und XXX.

So konnte nicht bei allen Klassifizierungsarten auf die einzelnen Erfolgsfaktoren in den jeweiligen Gruppen eingegangen werden, da
dies den umfang dieser Arbeit überschritten hätte. In der Übersicht jedoch wurde versucht, möglichst alle relevanten Informationen darzustellen.
%--------------Literaturverzeichnis-------------------------------------------------
\clearpage
\phantomsection
\addcontentsline{toc}{section}{Literaturverzeichnis}
\printbibliography[title=Literaturverzeichnis]

%--------------Erklärung-------------------------------------------------
\clearpage
\phantomsection
\addcontentsline{toc}{section}{Erklärung}
\section*{Erklärung}
\section*{Erklärung}
\addcontentsline{toc}{section}{Erklärung}

Hiermit versichere ich an Eides Statt, dass ich die vorliegende Arbeit selbstständig und ohne die Benutzung anderer als der angegebenen Hilfsmittel angefertigt habe. 
Alle Stellen, die wörtlich oder sinngemäß aus veröffentlichten und nicht veröffentlichten Schriften entnommen wurden, sind als solche kenntlich gemacht. 
Die Arbeit ist in gleicher oder ähnlicher Form oder auszugsweise im Rahmen einer anderen Prüfung noch nicht vorgelegt worden.

\vspace{30mm}
\begin{flushleft}
    Köln, den 27. Februar 2013
\end{flushleft}

\end{document}
