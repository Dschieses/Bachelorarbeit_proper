\documentclass[12pt,a4paper,oneside]{article}
%deutscher Zeichensatz
\usepackage[utf8]{inputenc}
%Brüche darstellen
\usepackage{nicefrac}
% deutsche Silbentrennung
\usepackage[ngerman]{babel} 
%Lange Tabellen
\usepackage{longtable}
%Lorem Ipsum
\usepackage{lipsum}
%Farben
\usepackage{xcolor}
%ToDO
\newcommand{\todo}[1]{\textbf{\textsc{\textcolor{red}{(TODO: #1)}}}}
%Margins
\usepackage[top=2.5 cm, bottom=2.5 cm, left=5 cm, right=1 cm]{geometry}
% Sonderzeichen, wie Eurosymbol
\usepackage{textcomp}

%Format Überschriften
\usepackage{titlesec}
\titleformat{\section}{\bfseries}{\thesection.\quad}{12pt}{}
\titleformat{\subsection}{\bfseries}{\thesubsection\quad}{12pt}{}

%Linien in Tabellen
\usepackage{booktabs}

%Hyperlinks usw     
\usepackage{url}
\urlstyle{rm}
% Literaturverzeichnis
\usepackage[isbn=false,citestyle=authoryear-ibid,bibencoding=latin1,hyperref=true,bibstyle=authortitle,backend=bibtex,maxbibnames=30]{biblatex}
\usepackage[babel,german=guillemets]{csquotes}
\usepackage[colorlinks=false, pdfborder={0 0 0}]{hyperref}
\addbibresource{bib/Literaturverzeichnis.bib}
\let\cite\textcite
%spezielles Abbildungsnummerierungszeugs
\usepackage{chngcntr}
\counterwithin{figure}{section}
%Abkürzungsverzeichnis
\usepackage{acronym}
%Sprache Times New Roman
%\usepackage{mathptmx}
\usepackage{txfonts} %Schriftart Times New Roman
%Sprache: Deutsch
\selectlanguage{ngerman}

%Zeilenabstand
\usepackage{setspace}

%Formatierung Inhaltsverzeichnis
\usepackage{tocloft, titletoc}  
\renewcommand{\cftsecleader}{\cftdotfill{\cftdotsep}}
\setlength{\cftaftertoctitleskip}{0em}

\renewcommand{\cfttoctitlefont}{\bfseries}
\tocloftpagestyle{fancy}   
\titlecontents{section} % set formatting for \section -
                        % \subsection must be formatted separately 
[2.3em]                 % adjust left margin
{\vspace{0.5em}}             % font formatting
{\hspace{-1.8em}.\contentslabel{0.7em}\hspace{1em}} % section label and offset
{\hspace*{-2.3em}}
{\titlerule*[1mm]{.}\contentspage}
\titlecontents{subsection} % set formatting for \subsection -
[3em]                 % adjust left margin
{\vspace{0.5em}}             % font formatting
{\contentslabel{2.3em}} % section label and offset
{\hspace*{-2.3em}}
{\titlerule*[1mm]{.}\contentspage}
\titlecontents{subsubsection} % set formatting for \subsubsection -
[4.0em]                 % adjust left margin
{\vspace{0.5em}}             % font formatting
{\contentslabel{2.3em}\hspace{0.8em}} % section label and offset
{\hspace*{-2.3em}}
{\titlerule*[1mm]{.}\contentspage}

%Kopfzeilen
\usepackage{fancyhdr} %Paket laden
\pagestyle{fancy}
\fancyhf{}
\renewcommand{\headrulewidth}{0.0pt} %obere Trennlinie
\fancyhead[C]{\nouppercase{\color{gray}{\thepage}}}
%Fußzeilen
\usepackage[hang,flushmargin]{footmisc}    
\renewcommand*{\footnotelayout}{\footnotesize} % size of text
\renewcommand{\footnotemargin}{1.2em} % margin between text and number
\setlength{\footnotesep}{1.5em}    % space between footnotes
\setlength{\skip\footins}{2.5em}   % space between text & footnotes
\begin{document}

\vspace*{1mm}

% Name
\thispagestyle{empty}
Peter Praeder

\vspace*{23mm}

% Bacheloararbeit
\begin{center}
\textbf{
Bacheloararbeit\linebreak
im Fach Allgemeine Wirtschaftsinformatik}
\end{center}

\vspace*{20mm}

% Titel
\begin{center}
\LARGE 
    Arten der Klassifizierung von Erfolgsfaktoren in IS-Projekten
\end{center}

\vspace*{8mm}

% Themensteller
\begin{center}
    Themensteller: Univ.-Prof. Dr. Werner Mellis
\end{center}

\vspace*{12mm}

% Vorgelegt
\begin{center}
    Vorgelegt in der Bachelorprüfung
\linebreak
    im Studiengang Wirtschaftsinformatik
\linebreak
    der Wirtschafts- und Sozialwissenschaftlichen Fakultät
\linebreak
    der Universität zu Köln
\end{center}
\vspace*{30mm}

% Köln, April 2013
\begin{center}
Köln, April 2013
\end{center}
\newpage


\pagenumbering{Roman}

\setcounter{page}{2}
\tableofcontents{}
\clearpage
%Abkürzungsverzeichnis
\addcontentsline{toc}{section}{Abkürzungsverzeichnis}
\section*{Abkürzungsverzeichnis}
\renewcommand{\bflabel}[1]{{#1\hfill}}
\begin{acronym}[Kernbegriffe]
 \acro{EF}{Erfolgsfaktor}
 \acro{IS}{Informationssystem}
 \acro{IT}{Informationstechnologie}
\end{acronym}
\newpage
%Abbildungsverzeichnis
%\addcontentsline{toc}{section}{Abbildungsverzeichnis}\listoffigures

%\clearpage

%Tabellenverzeichnis
%\addcontentsline{toc}{section}{Tabellenverzeichnis}\listoftables
%%\newpage
\onehalfspacing

%Zeilennummerierung arabisch
\pagenumbering{arabic}
% Einleitung
\section{Problemstellung}
Bereits seit mehreren Jahrzehnten werden die Faktoren, welche den Erfolg von IS-Projekten positiv beeinflussen, untersucht.\footnote{fortune white 2006}
Dabei wurden unterschiedliche Arten gefunden, diese \ac{EF} in Gruppen einzuordnen und somit zu klassifizieren.
Daraus ergibt sich ein theoretisches Erkenntnissdefizit, da es keine einheitliche, anerkannte Liste von \ac{EF} gibt. Dies kann wiederum dazu führen, dass
die Häufigkeit, mit der Projekte scheitern oder aus dem Zeit- bzw. Kostenrahmen laufen, steigt, da den Projektverantwortlichen die \ac{EF} nicht ausreichend bekannt sind.\footnote{Sudhakar S. 538}\todo{sinn rein bringen!}
Ein Problem für die Forschung ergibt sich aus der, in der Fachliteratur fehlenden, umfassenden Übersicht über diese Klassifizierungsarten, die die verschiedenen Autoren liefern.
Dieser Mangel erschwert die strukturierte Erfassung der maßgeblichen Komponenten, die das Gelingen eines IS-Projektes positiv beeinflussen und begünstigt somit die Wahrscheinlichkeit des Scheiterns dieser Projekte.
Der Mehrwert einer solchen Zusammenfassung besteht in der Möglichkeit, die Ergebnisse einer retrospektiven Analyse und die Gründe des Projekterfolgs leichter einzuordnen.
\\Zur Erstellung einer oben beschriebenen Übersicht muss man sich die folgende Frage stellen:
\textit{Welche unterschiedlichen Klassifizierungsarten von Erfolgsfaktoren von IS-Projekten werden in der Fachliteratur aufgezeigt und wie werden die Ansätze begründet?}\\
Hieraus definiert sich das Forschungsproblem:
Die weitere Forschung im Bereich der Erfolgsfaktorenklassifizierung gestaltet sich schwierig, da
verschiedene, uneinheitliche Ansätze der Klassifizierung von Erfolgsfaktoren existieren und es darüber hinaus keine Übersicht über die bereits \todo{Nomen, vorhandenen klein} Vorhandenen gibt.
Ein systematisches Literraturreview kann dazu beitragen, die bereits bestehenden Klassifizierungen zu extrahieren 
und eventuell Defizite oder Forschungslücken aufzeigen.
Eine aus diesem Review resultierende Übersicht über die in der Fachliteratur aufgeführten \KAS von 
Erfolgsfaktoren in IS-Projekten würde zur Lösung des Forschungsproblems beitragen.
\section{Ziel der Arbeit}
Das Hauptziel dieser Bachelorarbeit ist es, der Unübersichtlichkeit der Klassifizierungsarten von Erfolgsfaktoren in IS-Pojekten entgegenzuwirken.\\
Dazu ist zunächst zu klären, welche Arten in der Fachliteratur aufgezeigt werden. Zu\-sätz\-lich gilt es aufzuschlüsseln, welche Erklärungen die
jeweiligen Autoren für ihre Ansätze anbringen.
\section{Begriffsklärung}
In der Arbeit werden nicht als allgemein bekannt vorauszusetzende Begriffe verwendet, die im Folgenden definiert werden.
\subsection{Informationssystem}
Ein Informationssystem (IS) ist ein System, welches in die Organisations-, Personal- und Technikstrukturen eines Unternehmens eingebunden ist.\footnote{Vgl. zu diesem Absatz \cite{Laudon.2009} S.17.} 
Es wird speziell für Zwecke eines bestimmten Unternehmens(teils) entwickelt und implementiert.
Zudem enthält es die dazu benötigte Anwendungssoftware und Daten.
\subsection{Projekt}
Unter dem Begriff Projekt versteht man ein "`Vorhaben, das im Wesentlichen durch die Einmaligkeit der Bedingungen in ihrer
Gesamtheit gekennzeichnet ist, wie z.B. Zielvorgabe, zeitliche, finanzielle,
personelle und andere Begrenzungen; Abgrenzung gegenüber anderen Vorhaben;
projektspezifische Organisation"\footnote{\cite{DIN.200901}.}
, beziehungsweise ein zeitlich definiertes Vorhaben, das unternommen wird, um eindeutige Produkte, Dienstleistungen oder Ergebnisse zu erstellen.\footnote{Vgl. \cite{ProjectManagementInstitute.2008} S.5.}
\subsection{Erfolgsfaktor}
Unter Erfolgsfaktoren versteht man im allgemeinen Schlüsselfaktoren, die den Erfolgs eines Projektes fördern,\footnote{Vgl. \cite{Buschermohle.2010} zitiert nach \cite{Basten.2012}.} wie
z.B. Führungskompetenz und Erfahrung des Projektleiters, Kommunikation im Team oder Ünterstützung des Managements.\\
Oft wird auch von kritischen Erfolgsfaktoren gesprochen, wobei es für diesen Begriff keine allgemein anerkannte Definition gibt.\footnote{Vlg. \cite{Basten.2012}, S. 59.}

\subsection{Klassifizierung}
Klassifizierung beschreibt den Prozess der systematischen Zuweisung ähnlicher Objekte zu Objektklassen.\footnote{Vgl. \cite{Elmasri.2009} S. 118.}
\section{Vorgehensweise}
Um einen Überblick über die in der Fachliteratur verwendeten Klassifikationen von \EF in IS-Projekten zu schaffen, wurde ein 
systematisches Literaturreview in den Datenbanken von AIS Electronic Library (AISeL), 
EBSCOhost("`Academic Search Complete” und "`Business Source Complete"), ProQuest und ScienceDirect durchgeführt.
Dabei wurden jeweils die Titel, Schlagwörter und Abstracts nach den Begriffen 
"`Erfolgsfaktor"', "`Projekt"' und "`Informationssystem"', welche mit einem logischen UND verknüpft waren, durchsucht. 
Hierbei mussten die verschiedenen englischen Schreibweisen der Begriffe verwendet, sowie Plural und Synonyme abgedeckt werden.
Zum Beispiel wurden als Synonyme für Informationssystem auch die Begriffe "`Software"' und "`Informationstechnologie"' bzw. "`Information Technology"' verwendet.\\
Die gefundenen Ergebnisse wurden dann in einer Tabelle redundanzfrei festgehalten und einem weiteren Auswahlprozess unterzogen:
Nur Literatur, die sich schwerpunktmäßig mit \EF und deren Klassifizierung beschäftigt, sollte weiter betrachtet werden.
Dies wurde durch das Lesen des Abstracts festgestellt. Bei Unklarheiten wurden zusätzlich einzelne Textabschnitte,
vorrangig z.B. die Einleitung oder das Fazit, berücksichtigt.
Darauf folgte ein intensives Studium der verbliebenen Texte, in welchem die von den Autoren aufgezeigten Klassifizierungsarten und Begründungen extrahiert wurden.
\todo{In diesen Texten zitierte, relevante, jedoch noch nicht berücksichtigte Literatur wurde zusätzlich in das Studium aufgenommen.}
\section{Gliederung}
Im Folgenden ist der Aufbau der Bachelorarbeit aufgelistet. Neben der Kommentierung der einzelnen Kapitel ist hier auch der erwartete Umfang angegeben.\\
\\
\begin{longtable}{l p{17em}}
\textbf{1. Einleitung} & Die Einleitung umfasst Problemstellung, Zielsetzung, Vorgehensweise und Aufbau der Arbeit. (2-3 Seiten)\\
\midrule
\textbf{1.1 Problemstellung} & Mit welchem Problem befasst sich die Bachelorarbeit? Welche Relevanz haben diese Probleme und welche Antwort lässt sich finden? (\nicefrac{3}{4} Seite)\\
\midrule
\textbf{1.2 Zielsetzung} & Welches Ziel verfolgt die Bachelorarbeit? (\nicefrac{1}{4} Seite)\\
\midrule
\textbf{1.3 Vorgehensweise} & Wie wurde zur Zielerreichung der Arbeit vorgegangen? (\nicefrac{3}{4} Seite)\\
\midrule
\textbf{1.4 Aufbau der Arbeit} & Wie ist die Arbeit aufgebaut, sprich was ist im jeweiligen Kapitel der Arbeit zu finden? (\nicefrac{1}{2} Seite)\\
\toprule
\textbf{2.Grundlagen}  & Welche Grundlagen müssen geschaffen werden, damit ein Außenstehender diese Arbeit verstehen kann? (1-2 Seiten) \\
\midrule
\textbf{2.1 Definitionen} & Welche, nicht allgemein bekannten, Begriffe müssen zum Verständnis der Arbeit definiert werden? (1 Seite)\\
\toprule
\textbf{3. Analyse der Klassifizierungsarten} & Das Hauptkapitel befasst sich mit der Frage, welche Klassifizierungsarten in der Literatur adressiert werden. Welche Erklärungen für die jeweiligen Ansätze liefern die Autoren? (20-35 Seiten)\\
\toprule
\textbf{4.Fazit} & In welchem Umfang wurde das Ziel der Arbeit erreicht? Welches Fazit kann aus den Erkenntnissen der Arbeit gezogen werden? (1-2 Seiten) \\
\end{longtable}
\section{Erwartete Ergebnisse}
Das zu erwartende Ergebnis der Arbeit, bezogen auf das Hauptziel, ist eine Reduktion der Unübersichtlichkeit der Klassifizierungsarten von Erfolgsfaktoren in IS-Projekten.
Die, durch das systematische Literraturreview erstellte, Übersicht liefert einen einfachen und schnellen Überblick über die vorhandenen Arten und erleichtert das Erkennen von Forschungslücken und Defiziten,
zu deren Lösung dann weiter geforscht werden kann.
\section{Offene Punkte und Probleme}
%Durch eine sehr große Trefferzahl bei den Ergebnissen der Suche in den Datenbanken\todo{mehr}
Bis zum Zeitpunkt der Erstellung des Exposés ergaben sich ansonsten keine konkreten Probleme oder offenen Punkte.
\clearpage
%Literaturverzeichnis
\addcontentsline{toc}{section}{Literaturverzeichnis}
\printbibliography[title={Literaturverzeichnis}]

\end{document}
