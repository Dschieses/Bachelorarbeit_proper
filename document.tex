\documentclass[12pt,a4paper,oneside]{article}

%deutscher Zeichensatz
\usepackage[utf8]{inputenc}
\usepackage[T1]{fontenc}

%Brüche darstellen
\usepackage{nicefrac}

% deutsche Silbentrennung
\usepackage[ngerman]{babel}

%Farben
\usepackage{xcolor}
 
%Lange Tabellen
\usepackage{longtable}

%Lorem Ipsum
\usepackage{lipsum}

%Margins
\usepackage[top=2.5 cm, bottom=2.5 cm, left=5 cm, right=1 cm]{geometry}

% Sonderzeichen, wie Eurosymbol
\usepackage{textcomp}

%Format Überschriften
\usepackage{titlesec}
\titleformat{\section}{\bfseries}{\thesection.\quad}{12pt}{}
\titleformat{\subsection}{\bfseries}{\thesubsection\quad}{12pt}{}

%Linien in Tabellen
\usepackage{booktabs}

%Hyperlinks usw     
\usepackage[hyphens]{url}
\urlstyle{rm}


% Literaturverzeichnis
\usepackage[isbn=false,citestyle=authoryear-ibid,bibencoding=latin1,hyperref=true,bibstyle=authortitle,backend=bibtex,maxbibnames=30]{biblatex}
\usepackage[babel,german=guillemets]{csquotes}
\usepackage[colorlinks=false,hyperindex,pdfauthor={Peter Praeder},pdfborder={0 0 0}, pdfsubject={Bachelorarbeit},linkcolor={blue},citecolor={green},plainpages=false]{hyperref}
\addbibresource{bib/Literaturverzeichnis.bib}
\let\cite\textcite

%spezielles Abbildungsnummerierungszeugs
\usepackage{chngcntr}
\counterwithin{figure}{section}

%Abkürzungsverzeichnis
\usepackage[printonlyused]{acronym}    
\renewcommand{\bflabel}[1]{{#1\hfill}}

%Sprache Times New Roman
%\usepackage{mathptmx}
\usepackage{txfonts} %Schriftart Times New Roman
%Sprache: Deutsch
\selectlanguage{ngerman}

%Zeilenabstand
\usepackage{setspace}

% TOC, LOF, FIG Styles
\usepackage{tocloft, titletoc}  
\tocloftpagestyle{fancy}
\setlength{\cftaftertoctitleskip}{0em}
\renewcommand{\cftloftitlefont}{\bfseries}
%\renewcommand{\cftfigfont}{\bfseries}
\renewcommand{\cfttoctitlefont}{\bfseries}
\renewcommand{\cftlottitlefont}{\bfseries}
\titlecontents{section}     % set formatting for \section 
[2.3em]                     % adjust left margin
{\vspace{0.5em}}            % font formatting
{\hspace{-1.8em}.\contentslabel{0.7em}\hspace{1em}} % section label and offset
{\hspace*{-2.3em}}
{\titlerule*[1mm]{.}\contentspage}

\titlecontents{subsection}  % set formatting for \subsection 
[3em]                       % adjust left margin
{\vspace{0.5em}}            % font formatting
{\contentslabel{2.3em}}     % section label and offset
{\hspace*{-2.3em}}
{\titlerule*[1mm]{.}\contentspage}

\titlecontents{subsubsection}  % set formatting for \subsubsection 
[3.7em]                       % adjust left margin
{\vspace{0.5em}}            % font formatting
{\contentslabel{2.3em}}     % section label and offset
{\hspace*{-2.3em}}
{\titlerule*[1mm]{.}\contentspage}

\titlecontents{figure}      % set formatting for \subsection 
[2.3em]                     % adjust left margin
{\vspace{0.5em}}            % font formatting
{\contentslabel{2.3em}}     % section label and offset
{\hspace*{-2.3em}}
{\titlerule*[1mm]{.}\contentspage}

\titlecontents{table}       % set formatting for \subsection 
[2.3em]                     % adjust left margin
{\vspace{0.5em}}            % font formatting
{\contentslabel{2.3em}}     % section label and offset
{\hspace*{-2.3em}}
{\titlerule*[1mm]{.}\contentspage}


%Kopfzeilen
\usepackage{fancyhdr} %Paket laden



%Fußzeilen
\usepackage[hang,flushmargin]{footmisc}    
\renewcommand*{\footnotelayout}{\footnotesize} % size of text
\renewcommand{\footnotemargin}{1.2em} % margin between text and number
\setlength{\footnotesep}{1.5em}    % space between footnotes
\setlength{\skip\footins}{2.5em}   % space between text & footnotes

%--------------Neue Kommandos-------------------------------------------
\newcommand{\KA}{Klassifizierungsart }
\newcommand{\KAS}{Klassifizierungsarten }
\newcommand{\EF}{Erfolgsfaktoren }
\newcommand{\ISP}{IS-Projekt }
\newcommand{\ISPS}{IS-Projekten }
\newcommand{\BA}{Bachelorarbeit }

\newcommand{\todo}[1]{\textbf{\textsc{\textcolor{red}{(TODO: #1)}}}}
\newcommand{\syn}[1]{\textbf{\textsc{\textcolor{red}{(SYNONYM: #1)}}}}
%--------------Worttrennung-------------------------------------------
\hyphenation{be-rück-sich-tigt IS--Pro-jekt pro-jekt-er-folg}

%--------------DOCUMENT-------------------------------------------------
\begin{document}
\pagestyle{plain}
\vspace*{1mm}

% Name
\thispagestyle{empty}
Peter Praeder

\vspace*{23mm}

% Bacheloararbeit
\begin{center}
\textbf{
Bacheloararbeit\linebreak
im Fach Allgemeine Wirtschaftsinformatik}
\end{center}

\vspace*{20mm}

% Titel
\begin{center}
\LARGE 
    Arten der Klassifizierung von Erfolgsfaktoren in IS-Projekten
\end{center}

\vspace*{8mm}

% Themensteller
\begin{center}
    Themensteller: Univ.-Prof. Dr. Werner Mellis
\end{center}

\vspace*{12mm}

% Vorgelegt
\begin{center}
    Vorgelegt in der Bachelorprüfung
\linebreak
    im Studiengang Wirtschaftsinformatik
\linebreak
    der Wirtschafts- und Sozialwissenschaftlichen Fakultät
\linebreak
    der Universität zu Köln
\end{center}
\vspace*{30mm}

% Köln, April 2013
\begin{center}
Köln, April 2013
\end{center}
\newpage


\pagenumbering{Roman}

\setcounter{page}{2}

\pagestyle{fancy}
\fancyhf{}
\renewcommand{\headrulewidth}{0.0pt} %obere Trennlinie
\fancyhead[C]{\nouppercase{\color{gray}{\thepage}}}
\tableofcontents{}

%--------------Abkürzungsverzeichnis-------------------------------------------------
\clearpage
\phantomsection
\addcontentsline{toc}{section}{Abkürzungsverzeichnis}
\section*{Abkürzungsverzeichnis}
\section*{Abkürzungsverzeichnis}
\renewcommand{\bflabel}[1]{{#1\hfill}}
\begin{acronym}[Kernbegriffe]
 \acro{EF}{Erfolgsfaktor}
 \acro{IS}{Informationssystem}
 \acro{IT}{Informationstechnologie}
\end{acronym}

%--------------Abbildungsverzeichnis-------------------------------------------------
\clearpage
\phantomsection
\addcontentsline{toc}{section}{Abbildungsverzeichnis}
\listoffigures

%--------------Tabellenverzeichnis-------------------------------------------------
\clearpage
\phantomsection
\addcontentsline{toc}{section}{Tabellenverzeichnis}
\listoftables
\clearpage
\onehalfspacing

%--------------Zeilennummerierung arabisch-------------------------------------------------
\pagenumbering{arabic}
%--------------Einleitung-------------------------------------------------
\section{Einleitung}
\subsection{Problemstellung}
Bereits seit mehreren Jahrzehnten werden die Faktoren, welche für den Erfolgs von IS-Pro\-jek\-ten verantwortlich sind, untersucht. Dabei wurden unterschiedliche Arten gefunden,
diese Erfolgsfaktoren in Gruppen einzuordnen, sprich zu klassifizieren.
Das ü\-ber\-ge\-ord\-ne\-te Forschungsproblem ergibt sich aus der, in der Fachliteratur fehlenden, umfassenden Ü\-ber\-sicht über diese Klassifizierungsarten, die die verschiedenen Autoren liefern.
Dieser Mangel erschwert die strukturierte Erfassung der maßgeblichen Komponenten, die das Gelingen eines IS-Projektes positiv beeinflussen und begünstigt somit die Wahrscheinlichkeit des Scheiterns dieser Projekte.
Der Mehrwert einer solchen Zusammenfassung besteht in der Möglichkeit, die Ergebnisse einer retrospektiven Analyse und die Gründe des Projekterfolgs leichter einzuordnen.
\\Zur Erstellung einer oben beschriebenen Übersicht muss man sich die folgende Frage stellen:
\textit{Welche unterschiedlichen Klassifizierungsarten von Erfolgsfaktoren von IS-Projekten werden in der Fachliteratur aufgezeigt und wie werden die Ansätze begründet?}\\
Hieraus definiert sich das eigentliche Forschungsproblem:
Die weitere Forschung im Bereich der Erfolgsfaktorenklassifizierung würde sich einfacher gestalten, wenn es eine Übersicht über die bereits vorherrschenden Klassifizierungsarten von Erfolgsfaktoren gäbe.
Ein systematisches Literraturreview kann dazu beitragen, die bereits bestehenden Klassifizierungen zu extrahieren und eventuell Defizite oder Forschungslücken aufzeigen.
Eine aus diesem Review resultierende Übersicht über die in der Fachliteratur aufgeführten Klassifizierungsarten von Erfolgsfaktoren in IS-Projekten würde zur Lösung des übergeordneten Forschungsproblems beitragen.
\subsection{Zielsetzung}
Das Hauptziel dieser Bachelorarbeit ist es, der Unübersichtlichkeit der Klassifizierungsarten von Erfolgsfaktoren in IS-Pojekten entgegenzuwirken.\\
Dazu ist zunächst zu klären, welche Arten in der Fachliteratur aufgezeigt werden. Zu\-sätz\-lich gilt es aufzuschlüsseln, welche Erklärungen die
jeweiligen Autoren für ihre Ansätze anbringen.
\subsection{Vorgehensweise}
Um einen Überblick über die in der Fachliteratur verwendeten Klassifikationen von Erfolgsfaktoren in IS-Projekten zu schaffen, wird ein 
systematisches Literaturreview durchgeführt. Dazu werden die Online-Portale von ACM digital library, AIS Electronic Library (AISeL), 
EBSCOhost("`Academic Search Complete” und "`Business Source Complete"), EmeraldInsight, IEEEXplore, ProQuest, ScienceDirect, SpringerLink und 
Wiley InterScience nach relevanten Texten durchsucht. Die Titel, Schlagwörter oder Zusammenfassungen der Literatur
sollen eine logische Verknüpfung der folgenden Begriffe enthalten: Erfolgsfaktor, Projekt, Software, Informationssystem und Informationstechnologie. 
Dabei muss darauf geachtet werden, dass sowohl die verschiedenen englischen Schreibweisen der Begriffe abgedeckt werden, als auch, dass „Success Factor“ und „Project“ und
mindestens einer der Begriffe „Software“, “Information System“ oder „Information Technology“ enthalten ist.\\ 
Die gefundenen Ergebnisse werden dann in einer Tabelle redundanzfrei festgehalten. Beim Lesen der Abstracts und gegebenenfalls der Texte wird festgestellt, ob diese sich tatsächlich auf den 
gewünschten Sachverhalt beziehen. Suchergebnisse, die in keinerlei inhaltlichem Zusammenhang zur Thematik stehen, werden hierbei verworfen.
Sollte in diesen Texten auf noch nicht berücksichtigte, relevante Literatur verwiesen werden, so wird diese nachgetragen.
Darauf folgt ein intensives Studium der relevanten Texte, in welchem die von den Autoren aufgezeigten Klassifizierungsarten und Begründungen extrahiert werden.



\subsection{Aufbau der Arbeit}
Im folgenden Kapitel werden zunächst elementare Grundlagen vermittelt, wie die Definition von Begriffen, die für das Verständnis dieser Arbeit wichtig sind.
Darauf folgt im dritten Kapitel der Schwerpunkt der Arbeit. Basierend auf dem Literraturreview werden die verschiedene Klassifizierungsarten dargestellt und zum Schluss zu einer 
Übersicht gruppiert.\todo{schöner}

blablabla autoren chronologisch, da sie gegebenenfalls aufeinander aufbauen
%--------------Grundlagen-------------------------------------------------
\section{Grundlagen}
\subsection{Definitionen}

\subsubsection{Informationssystem}
Ein Informationssystem (IS) ist ein System, welches in die Organisations-, Personal- und Technikstrukturen eines Unternehmens eingebunden ist.\footnote{Vgl. zu diesem Absatz \cite{Laudon.2009} S.17.} 
Es wird speziell für Zwecke eines bestimmten Unternehmens(teils) entwickelt und implementiert.
Zudem enthält es die dazu benötigte Anwendungssoftware und Daten.

\subsubsection{Projekt}
Unter dem Begriff Projekt versteht man ein zeitlich definiertes Vorhaben, welches unternommen wird,
um eindeutige Produkte, Dienstleistungen oder Ergebnisse zu erstellen.\footnote{Vgl. \cite{ProjectManagementInstitute.2008} S.5.}
Ein Projekt ist im Wesentlichen durch die Einmaligkeit der Bedingungen in ihrer Gesamtheit gekennzeichnet, wie z.B. Zielvorgaben, zeitliche, finanzielle und personelle Begrenzungen.\footnote{Vgl. \cite{DIN.200901} S.11.}

\subsubsection{Projekterfolg}
Projekterfolg ist das zusammenfassende Ergebnis der Beurteilung des Projekts hinsichtlich der Zielerreichung.\footnote{Vgl. \cite{DIN.200901} S.13.}
Neben den objektiv messbaren Zielkriterien, wie Ergebnis, Termintreue oder Budgettreue, hängt die Beutreilung des Projekterfogs auch von Standpunkt des jeweiligen Stakeholders ab.\footnote{Vgl. für diesen und den nächsten Satz \cite{Angermeier.o.J.}}
Zum Beispiel kann auch die Zufriedenheit des Auftraggebers oder die Bezahlung der Schlussrechnung als Kriterium für den Projekterfolg herangezogen werden. 

\subsubsection{Erfolgsfaktoren}
Erfolgsfaktoren im Alllgemeinen sind eine limitierte Anzahl von Faktoren, deren zufriedenstellendes Ergebnis erfolgreiche und 
wettbewerbsfähige Leistung für einzelne Bereiche oder das ganze Unternehmen sicherstellen.\footnote{Vgl. \cite{Bullen.1981} S.7.}
Bezogen auf Projekte sind dies Schlüsselfaktoren, die den Erfolg des Projektes fördern,\footnote{Vgl. \cite{Buschermohle.2010} zitiert nach \cite{Basten.2012}.} wie
z.B. Führungskompetenz und Erfahrung des Projektleiters, Kommunikation im Team oder Unterstützung des Managements.\\
Oft wird auch von kritischen Erfolgsfaktoren gesprochen, wobei es für diesen Begriff keine allgemein anerkannte Definition gibt.\footnote{Vlg. \cite{Basten.2012}, S. 59.}

\subsubsection{Klassifizierung}
Klassifizierung beschreibt den Prozess der systematischen Zuweisung ähnlicher Objekte zu Objektklassen.\footnote{Vgl. \cite{Elmasri.2009}, S. 118.}
Bezogen auf \EF ist eine \KA demnach die Einordnung von \EF in Kategorien.


\subsubsection{Klassifizierungsart}


\subsection{Klassifizierung von \EF}
Rockart\footnote{\cite{Rockart.1979}} entwickelte als einer der Ersten den Ansatz, Erfolgsfaktoren zu identifizieren und Unternehmensleistung zu messen.\footnote{Vgl. für diesen und den folgenden Satz \cite{Chow.2008}, S. 962.}
Bullen und Rockart\footnote{\cite{Bullen.1981}} sowie Rockart und Crescenzi\footnote{\cite{Rockart.1984}} etabliereten und verfeinerten den Ansatz daraufhin.\\
Seit Beginn der Forschung zu Erfolgsfaktoren von Projekten, wurden immer wieder Auflistungen von Faktoren 
geliefert, jedoch wurde nur eine geringe Priorität auf das Klassifizieren von EF gelegt.\\\noindent
Der Vorteil einer solchen Einordnung in Gruppen liegt darin, dass man, anstatt einzelne Erfolgsfaktoren zu analysieren, zunächst die Gruppen identifizieren kann, in die die Faktoren
einzuordnen sind.\footnote{Vgl. zu diesem und dem folgenden Satz \cite{Belassi.1996}, S. 142.} Daraufhin können dann die kombinierten Auswirkungen der Faktoren auf den Projekterfolg zu bestimmen.

Die Klassifizierung von EF kann jedoch bei der Analyse der Wechselwirkung zwischen diesen und deren Konsequenzen helfen.\footnote{Vgl. \cite{Belassi.1996}, S. 142.} 



%\textit{Motivation:}\\\noindent
%Die Motivation und das formulierte Problem aus Sicht der Autoren liegt in der Ambiguität der Beurteilung von Projekterfolg.S141-142
%Für diese sehen die Autoren zwei Gründe: Der erste und untergeordnete Grund basiert auf der subjektiven Einschätzung des 
%Projektergebnisses. Auch Projekte, die das Management als gescheitert ansieht, können für den Kunden ein Erfolg sein (ZITAT)141.
%Der primäre Problemansatz liegt aber in der starken Variation der Auflistungen von EF in der Fachliteratur. Dies begünstigt (??) den 
%generellen Trend zur Einordnung von EF in Tabellen, anstatt diese strukturiert bestimmten Kriterien zuzuordnen und so zu kategorisieren. 
%Weitgehend sind es oftmals nicht nur die einzelnen EF, die einen großen Einfluss auf den Erfolg eines Projektes haben, sondern auch das Zusammenspiel 
%von mehreren Faktoren aus unterschiedlichen Kategorien und in unterschiedlichen Phasen des Projektes.\\
\subsubsection{Warum}
\subsubsection{Wie}
\subsubsection{Besser?}


%--------------Hauptteil-------------------------------------------------
\clearpage
\section{Klassifizierungsarten in der Literatur}
\subsection{Klassifizierungsarten und deren Begründungen}
\subsubsection{Studie a}

\subsection{Übersicht}

\subsection{Schnittmengenbetrachtung}
\subsubsection{Konstante \EF}
\subsubsection{Klassifizierung der konstanten \EF}

%--------------Fazit-------------------------------------------------
\section{Fazit}

%--------------Literaturverzeichnis-------------------------------------------------
\clearpage
\phantomsection
\addcontentsline{toc}{section}{Literaturverzeichnis}
\printbibliography[title=Literaturverzeichnis]

%--------------Erklärung-------------------------------------------------
\clearpage
\phantomsection
\addcontentsline{toc}{section}{Erklärung}
\section*{Erklärung}
\section*{Erklärung}
\addcontentsline{toc}{section}{Erklärung}

Hiermit versichere ich an Eides Statt, dass ich die vorliegende Arbeit selbstständig und ohne die Benutzung anderer als der angegebenen Hilfsmittel angefertigt habe. 
Alle Stellen, die wörtlich oder sinngemäß aus veröffentlichten und nicht veröffentlichten Schriften entnommen wurden, sind als solche kenntlich gemacht. 
Die Arbeit ist in gleicher oder ähnlicher Form oder auszugsweise im Rahmen einer anderen Prüfung noch nicht vorgelegt worden.

\vspace{30mm}
\begin{flushleft}
    Köln, den 27. Februar 2013
\end{flushleft}
\end{document}
